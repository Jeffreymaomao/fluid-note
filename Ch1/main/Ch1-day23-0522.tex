Now, the $z-z$ component of the inertia tensor is
\begin{equation}
\begin{aligned}
I_{33}
&= \int_{-\delta z/2}^{\delta z/2}\int_{-\delta y/2}^{\delta y/2}\int_{-\delta x/2}^{\delta x/2} \rho\left(\bar{x}^2+\bar{y}^2\right)\,d\bar{x}\,d\bar{y}\,d\bar{z}\\
&= \rho \left(\frac{(\delta x)^3}{24}\left(\delta y\,\delta z\right) + \frac{(\delta z)^3}{24}\left(\delta x\,\delta z\right)\right)\\
&= \frac{\rho}{24}\left((\delta x)^2 + (\delta y)^2\right) \delta V
\end{aligned}
\end{equation}
where $\rho$ is the density $\rho = \rho(x,y,z)$ which is a constant in $G_1$

So the Angular acceleration
\begin{equation}
\alpha \approx \frac{N_3}{I_{33}} \sim \frac{(S_{12}-S_{21})\delta V}{\frac{\rho}{24}\left((\delta x)^2 + (\delta y)^2\right) \delta V} = \frac{24(S_{12} - S_{21})}{\rho \left((\delta x)^2 + (\delta y)^2\right)}
\end{equation}
Consider 
\begin{equation}
\lim_{\delta x,\delta y\to 0} \alpha \to \pm \infty,\quad\text{if $S_{12}\neq S_{21}$}
\end{equation}
So that $S_{12} = S_{21}$.

We now write
\begin{equation}
S_{ij} = -\delta_{ij} p + T_{ij},
\end{equation}
where $p$ is pressure and $T_{ij}$ is called shear stress tensor.

Note that $T_{ij} = T_{ji}$ is a symmetric tensor.

The force per unit volumn to the fluid at $(x,y,z)$ in the $j$-th direction is 
\begin{equation}
F_{j} = \operatorname{div}\begin{pmatrix}S_{1j}\\S_{2j}\\S_{3j}\end{pmatrix}
= \operatorname{div}\begin{pmatrix}
-\delta_{1j} p + T_{1j}\\
-\delta_{2j} p + T_{2j}\\
-\delta_{3j} p + T_{3j}
\end{pmatrix} = -\frac{\partial p}{\partial x_{j}} + \operatorname{div}\begin{pmatrix}T_{1j}\\T_{2j}\\T_{3j}\end{pmatrix}
\end{equation}
now the force to the fluid at $(x,y,z)$ is $-\nabla p + \nabla \cdot T$.

Then by an argument similar to the proof of the Euler equation
\begin{equation}
\frac{\partial \textbf{u}}{\partial t} + \left(\textbf{u}\cdot \nabla\right)\textbf{u} = -\frac{\nabla p}{\rho} + \frac{1}{\rho}\left(\nabla\cdot T + \rho \cdot \textbf{b}\right)
\end{equation}
where $\textbf{b}$ is the external force per unit mass.

\subsubsection{Definition} % H3 title

A Newtonian fluid is one in which the shear stress tensor is a linear function of the rate of strain tensor which describe the deivation of the fluid motion from a regid motion.

\subsubsection{Definition} % H3 title

The rate of strain tensor may be defined as that controlling the evolution of the relative position of points in a fluid element.

\vspace{5pt}
\hrule
\vspace{6pt}

To account for these motions, let $\delta x$ denote an infinitesmal displacement of $2$ points in the fluid, one at $x(t)$ and the other at $x(t)+\delta x(t)$, 

The rate of change of $\lVert \delta x(t)\rVert^2$ is 
\begin{equation}
\begin{aligned}
\frac{d}{dt}\left(\delta x(t)\cdot \delta x(t)\right)
&= 2\delta x(t) \cdot \frac{d}{dt}\delta x(t)\\
&= 2\delta x(t) \cdot \left[\textbf{u}(x(t) + \delta x(t)) - \textbf{u}(x(t))\right]\\
&= 2\delta x \cdot \left(\textbf{u}(x+\delta x) - u(x)\right)\\
&\approx 2\delta x_{i} \frac{\partial u_{i}}{\partial x_{j}}(x)\cdot \delta x_{j}
\end{aligned}
\end{equation}
Then
\begin{equation}
(MVT) = \delta x_{i}\left(\frac{\partial u_i}{\partial x_j} + \frac{\partial u_j}{\partial x_i}\right)\delta x_{j} = 2\delta x\cdot R\cdot \delta x,
\end{equation}
where $R = (r_{ij})$, and $\displaystyle r_{ij} = \left(\frac{\partial u_i}{\partial x_j} + \frac{\partial u_j}{\partial x_i}\right)$ is the rate of strain tensor, which is the symmetric part of the velocity gradient tensor.

\begin{quote}
	\textbf{Recall :}
$s_{ij} = -\delta_{ij} p + T_{ij}$ and Newtonian fluid is one for which $T_{ij}$ is a linear function,
\end{quote}

\subsubsection{Definition} % H3 title

\begin{enumerate}
	\item A property is said to be isotropy if the property is independent

\end{enumerate}

\subsubsubsection{1st rank tensor} % H4 title




