\subsubsection{Navier-Stoke's equation} % H3 title


\begin{equation}
\textbf{u}_{t} + \left(\textbf{u}\cdot\nabla\right)\textbf{u}
=
-\frac{\nabla p}{\rho}
+
\frac{1}{\rho} \left(\mu\nabla^2\textbf{u} + \left(\lambda+\frac{1}{3}\mu\right)\nabla\left(\operatorname{div}\textbf{u}\right)\right) + \vec{b}
\end{equation}

For incompressible homogeneous flow $\rho = \rho_0$ is a constant without external force $\vec{b} = 0$, we have
\begin{equation}
\rho_0\frac{D\textbf{u}}{Dt} = - \nabla p + \mu\nabla^2\textbf{u}
\end{equation}


\begin{itemize}
	\item $\ell$ is charasteristic length
	\item $c$ is charasteristic velocity
	\item $\tau = \ell/c$ is charasteristic time

\end{itemize}
We can then rescale $\textbf{x}$, $\textbf{u}$ and $t$ by $\ell$, $c$ and $\tau$, we let
\begin{equation}
\bar{\textbf{u}} = \frac{\textbf{u}}{c},\quad
\bar{\textbf{x}} = \frac{\textbf{x}}{\ell},\quad
\bar{t} = \frac{t}{\tau}
\end{equation}
then
\begin{equation}
\textbf{u} = c\,\bar{\textbf{u}},\quad
\textbf{x} = \ell\,\bar{\textbf{x}},\quad
t = \tau\,\bar{t}
\end{equation}



consider
\begin{equation}
\begin{aligned}
\frac{D\textbf{u}}{Dt}
&= \frac{\partial \textbf{u}}{\partial t} + \left(\textbf{u}\cdot\nabla\right)\textbf{u}\\
&= \frac{\partial (c\,\bar{\textbf{u}})}{\partial (\tau\,\bar{t})} + \left((c\,\bar{\textbf{u}})\cdot\nabla_{\ell\bar{\textbf{x}}}\right)(c\,\bar{\textbf{u}})\\
&= \frac{c}{\tau}\frac{\partial \bar{\textbf{u}}}{\partial \bar{t}} + \frac{c^2}{\ell}\left(\bar{\textbf{u}}\cdot\nabla_{\bar{\textbf{x}}}\right)\bar{\textbf{u}}\\
&= \frac{c^2}{\ell^2}\frac{\partial \bar{\textbf{u}}}{\partial \bar{t}} + \frac{c^2}{\ell}\left(\bar{\textbf{u}}\cdot\nabla_{\bar{\textbf{x}}}\right)\bar{\textbf{u}}\\
&= \frac{c^2}{\ell} \frac{D\bar{\textbf{u}}}{D\bar{t}}
\end{aligned}
\end{equation}
and
\begin{equation}
\nabla_{\textbf{x}}^2 \textbf{u} =\nabla_{\ell\textbf{x}}^2 (c\,\bar{\textbf{u}}) = \frac{c}{\ell^2} \bar{\textbf{u}}
\end{equation}
last 
\begin{equation}
\nabla_{\textbf{x}} p = \nabla_{\ell\bar{\textbf{x}}} p =\frac{1}{\ell} \nabla_{\bar{\textbf{x}}}p
\end{equation}
the Navier-Stoke's equation becomes
\begin{equation}
\begin{aligned}
\rho_0\frac{D\textbf{u}}{Dt}
&= - \rho_0\frac{\nabla_{\bar{\textbf{x}}} p}{\rho} + \mu\nabla_{\bar{\textbf{x}}}^2\textbf{u}\\
\rho_0\frac{c^2}{\ell}\frac{D\bar{\textbf{u}}}{D\bar{t}}
&= - \frac{1}{\ell}\nabla_{\bar{\textbf{x}}}p + \frac{c\mu}{\ell^2}\nabla_{\bar{\textbf{x}}}^2\bar{\textbf{u}}\\
\frac{D\bar{\textbf{u}}}{D\bar{t}}
&= -\frac{1}{\rho_0\ell} \nabla_{\bar{\textbf{x}}} p + \frac{\mu}{\rho_0}\frac{1}{\ell c}\nabla_{\bar{\textbf{x}}}^2\bar{\textbf{u}}
\end{aligned}
\end{equation}
then define the charasteristic pressure $\bar{p}$ by
\begin{equation}
\bar{p} = \frac{p}{\rho_0\ell},\quad \Rightarrow\quad
p = \rho_0\ell \bar{p}
\end{equation}
also define the Reynolds number $R$ by
\begin{equation}
R = \frac{\ell c}{\mu/\rho_0} = \frac{\ell c}{\nu}
\end{equation}
where $\nu$ is $\nu = \mu/\rho_0$. Last, we have
\begin{equation}
\frac{D\bar{\textbf{u}}}{D\bar{t}}
= - \bar{\nabla} \bar{p} + \frac{1}{R}\bar{\nabla}^2\bar{\textbf{u}}
\end{equation}
where $\bar{\nabla} = \nabla_{\bar{\textbf{x}}}$, and $\operatorname{div}\bar{\textbf{u}} = 0$.

\subsubsection{Definition} % H3 title

We define the Reynolds number $R$ to be the dimensionless number 
\begin{equation}
R = \frac{\ell c}{\nu}
\end{equation}
For example, consider $2$ flows past $2$ spheres centered at the origin but different radii, one with a fluid where $A_{\infty} = 10\,\mathrm{km/hr}$. past a sphere of radius $10\,\mathrm{m}$ and the with the same fluid with $A_{\infty} = 100\,\mathrm{km/hr}$ and radius $1\,\mathrm{m}$. If we choose $\ell$ to the radius of the sphere and $A$ to be the velocity $A_{\infty}$, the Reynolds unmber 
\begin{equation}
R = \frac{0.01\times 10}{\nu} = \frac{0.001\times 100}{\nu}
\end{equation}
is the same fro each flow.

So, the rescaled equation for each flow in the dimensionless variables are the same.

\subsubsection{Definition} % H3 title

$2$ flow with the same geometry and the same Reynolds number are called similar. More precisely, let $\textbf{u}_1$ and $\textbf{u}_2$ be $2$ flows on the region $D_1$ and $D_2$ that are related by a scale factor $\lambda$, s.t. $L_1 = \lambda L_2$, Let $c_1$, $c_2$ be the characteristic velocity for each flow, and $\nu_1$, $\nu_2$ be the respective viscosities. If
\begin{equation}
\frac{L_1u_1}{\nu_1} =\frac{L_2u_2}{\nu_2}
\end{equation}
then $R_1 = R_2$, and the dimensionless velocity filed $\bar{\textbf{u}}_1$ and $\bar{\textbf{u}}_2$ satiesfies exact the same eqaution on the same region, Thus $\bar{\textbf{u}}_1$ can be obtained by rescaling $\bar{\textbf{u}}_2$ suitably.

\subsection{Helmholtz-Hodge decomposition Theorem} % H2 title

A vector filed $W$ on $D\subset \mathbb{R}^2$ or $D\subset \mathbb{R}^3$ can be uniquely decomposes in the form 
\begin{equation}
W = \nu + \nabla q
\end{equation}
for some function $q$ uni
