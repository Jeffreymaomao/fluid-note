\subsubsubsection{1. proof: existence} % H4 title

By the theory on elliptic ODE, there exit a solution $q$ of 
\begin{equation}
\begin{cases}
\nabla^2 q = \operatorname{div} w\quad\text{in }D\\
\displaystyle \frac{\partial q}{\partial n}\bigg|_{\partial D} = (\nabla q)\cdot n = w\cdot n
\end{cases}
\end{equation}
Let $v=w-\nabla q$, then
\begin{equation}
\begin{cases}
\operatorname{div} v = \operatorname{div}w - \operatorname{div}\nabla q = \operatorname{div}w - \nabla^2 q = 0\\
\displaystyle v\cdot n = w\cdot n - \frac{\partial q}{\partial n}=0,\quad
\text{on $D$ and $w = v + \nabla q$}
\end{cases}
\end{equation}


\subsubsubsection{2. proof: uniqueness} % H4 title

\paragraph{Step1} % H5 title

\textbf{Claim}: For any vector field $V$ on $D$ satisfying $\displaystyle \begin{cases}\operatorname{div}v=0&\text{in $D$}\\ v\cdot n = 0&\text{on $\partial D$}\end{cases}$, and any function $q$ on $D$, we have $\displaystyle \int_{D} v\cdot \nabla q dx = 0$.

\textbf{proof of claim}: Since $\operatorname{div}(qv) = \nabla q \cdot v + q\cdot \operatorname{div}v = \nabla q \cdot v$, by the divergence theorem 
\begin{equation}
\int_{D}\nabla q \cdot v dx = \int_{D} \operatorname{div}(qv) dx = \int_{\partial D}(qv)\cdot n d\sigma = 0
\end{equation}


\paragraph{Step2} % H5 title

Suppose $w = v_{1} + \nabla q_1 = v_2 + \nabla q_2$ , where $v_1, v_2$ satisfies
\begin{equation}
\displaystyle \begin{cases}\operatorname{div}v_i=0&\text{in $D$}\\ v_i\cdot n = 0&\text{on $\partial D$}\end{cases},\quad \forall i=1,2
\end{equation}
then
\begin{equation}
\begin{cases}
(v_1 - v_2) + \nabla (q_1 - q_2) = 0\\
\left(v_1 - v_2\right) \cdot n = 0
\end{cases}
\end{equation}
By claim
\begin{equation}
\begin{aligned}
0 &=\int_{D}\left(v_1 - v_2\right) \cdot \nabla\left(q_1 - q_1\right) dx\\
&= \int_{D}\left(v_1 - v_2\right)\left[-\left(v_1 - v_2\right)\right] dx\\
&= -\int_{D}\left(v_1 - v_2\right)^2 dx\\
\end{aligned}
\end{equation}
we have $v_1 = v_2$.

\subsubsection{Definition} % H3 title

We define the operator $\mathbf{P}$, an orthogonal projection operator, which maps $w$ onto its divergence free part $v$ in the Helmoltz-Hodge decomposition, then
\begin{equation}
w = \mathbf{P}w + \nabla q
\end{equation}


\begin{quote}
	\textbf{Rmk :}
If $v$ satisfies $\displaystyle \begin{cases}\operatorname{div}v=0&\text{in $D$}\\ v\cdot n = 0&\text{on $\partial D$}\end{cases}$, then $\mathbf{P} v = v$
\end{quote}

\begin{quote}
	\textbf{Rmk :}
$\mathbf{P}(\nabla q) = 0$
\end{quote}

\begin{quote}
	\textbf{Rmk :}
Consider the NS equations
\begin{equation}
\begin{aligned}
\textbf{u}_t + \left(\textbf{u}\cdot \nabla \right) &= - \nabla p + \frac{1}{R} \nabla^2 \textbf{u}\\
\textbf{u}_t + \nabla p &= -\left(\textbf{u}\cdot \nabla \right) \textbf{u} + \frac{1}{R}\nabla^2\textbf{u}\\
\mathbf{P}\left(\textbf{u}_t + \nabla p\right)
&= \mathbf{P}\left(-\left(\textbf{u}\cdot \nabla \right) \textbf{u} + \frac{1}{R}\nabla^2\textbf{u}\right)\\
\end{aligned}
\end{equation}
Since $\mathbf{P}(\nabla p) = 0$, and
\begin{equation}
\begin{cases}\operatorname{div}\textbf{u}_t=0&\text{in $D$}\\ \textbf{u}_t\big|_{\partial D} = 0&\text{on $\partial D$}\end{cases}
\end{equation}
we have
\begin{equation}
\mathbf{P}(\textbf{u}_t) = \textbf{u}_t
\end{equation}
then
\begin{equation}
\textbf{u}_t = \mathbf{P}\left(-\left(\textbf{u}\cdot \nabla \right) \textbf{u} + \frac{1}{R}\nabla^2\textbf{u}\right)
\end{equation}

\end{quote}
\subsubsection{Definition} % H3 title

In the NS equation for a viscous incompressible fluid
\begin{equation}
\partial_{t} \textbf{u} + \left(\textbf{u}\cdot\nabla\right)\textbf{u} = -\nabla p + \frac{1}{R}\nabla^2 \textbf{u}
\end{equation}
where $\operatorname{div}\textbf{u} = 0$. The term $\nabla^2 \textbf{u}/R$ is the diffusion or dissipation term and $\left(\textbf{u}\cdot\nabla\right)\textbf{u}$ is the inertia or convective term.s

If we neglect the term $\left(\textbf{u}\cdot\nabla\right)\textbf{u}$, we get the Stoke's equation for incompressible flow, this \textit{parabolic type} equation and the solution of the Stoke's equation provide a good approximation to the solution of the NS equation.

\section{Potential Flow} % H1 title

\subsection{Potential flow theorey} % H2 title

Through out this section, all flows are understood to be ideal (inviscid), in the other words, the flow is either incompressible nonviscous ot isentropic and nonviscous.

\begin{quote}
	\textbf{Rmk :}
By the previews result, for ideal flow if the flow is irrotation at intial time then it will it will remain irrotational for all time.
\end{quote}

\subsubsection{Definition} % H3 title

An inviscid irrotational flow is called a potential flow.

\subsubsection{Definition} % H3 title

A domain $D$ is said to be simplify connected if for any continuous closed curve $\gamma :[a,b]\to D$, $\gamma(a)=\gamma(b)$, there exist a continuous function $F:[a,b]\times [0,1]\to D$, s.t.
\begin{equation}
\begin{aligned}
F(s,0) = \gamma(s),&\quad \forall s\in [a,b]\\
F(a,t) = F(b,t),&\quad \forall t\in [0,1]\\
F(s,1) = P_0,&\quad \forall s \in [a,b] \text{ for some $P_0 \in D$}
\end{aligned}
\end{equation}
If such $F$ exist, we say that the curve $\gamma$ can be continuously shrink to a point in $D$.
