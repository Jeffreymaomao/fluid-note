\underline{Fig}

For the flow
\begin{equation}
\textbf{u}(x,y,t) = \left(u_1(x,t),0,0\right)
\end{equation}
and the equation
\begin{equation}
\textbf{u}_t + (\textbf{u}\cdot \nabla)\textbf{u} = -\frac{\nabla p}{\rho}
\end{equation}
becomes
\begin{equation}
u_{1,t} + u_1\frac{\partial u_1}{\partial x} = -\frac{p_x}{\rho_{0}}
\quad\Rightarrow\quad
u_{1,t} + u_1u_{1,x} = -\frac{p_x}{\rho_{0}}
\end{equation}
with the incompressibility $\operatorname{div}\textbf{u} = u_{1,x} = 0$, we have 
\begin{equation}
u_{1,t} = -\frac{p_x}{\rho_{0}}
\quad\Rightarrow\quad
p_x = -\rho_{0} u_{1,t}
\end{equation}
Notice that
\begin{equation}
p_{xx} = -\rho_{0} u_{1,t,x} = -\rho_{0} \frac{\partial}{\partial t} u_{1,x} = 0
\end{equation}
so that the pressure $p$ is given by
\begin{equation}
p_{x,x} = 0
\quad\Rightarrow\quad
p_{x} = C_1
\quad\Rightarrow\quad
p(x) = C_1 x + C_2,
\end{equation}
where $C_1,C_2$ is constant.

Plugin the boundary condition
\begin{equation}
\begin{cases}
p(x=0) = p(0) = p_1 = C_2\\
p(x=L) = p(L) = p_2 = C_1L+p_1
\end{cases},\quad\Rightarrow\quad
C_2 = \frac{p_2-p_1}{L}
\end{equation}
solving that the pressure
\begin{equation}
p(x) = p_1 + \left(p_2-p_1\right)\frac{x}{L}
\end{equation}
or
\begin{equation}
p(x) = \frac{(1-x)p_1 + xp_2}{L}.
\end{equation}
Notice that $p_x = (p_2-p_1)/L$, we can solve that 
\begin{equation}
u_{1,t} = \frac{p_2-p_1}{L\rho}
\quad\Rightarrow\quad
u_{1}(t) = \frac{p_2-p_1}{L\rho} t + C_3,
\end{equation}
for some constant $C_3$.

\begin{quote}
	Rmk:
We can observe that, when $t\to\infty$
\begin{equation}
\lim_{t\to\infty} u_{1}(t) = \infty
\end{equation}
which is impossible in real flow. Thus the Euler equation is not a good model for this flow.
\end{quote}

This is because we have ignored frictional force in the modelling. This situation will be remedied by the "\textit{Navier-Stokes equation}", which take account for friction force later.

\subsection{Rotation and Vorticity} % H2 title

\begin{quote}
	\textbf{Definition}
If the velocity filed of a fluid is $\textbf{u}=(u_1,u_2,u_3)$. We define the vorticity of the fluid
\begin{equation}
\mathbf{\zeta} = \nabla \times \textbf{u}
= \begin{vmatrix}
\textbf{e}_1 & \textbf{e}_2 & \textbf{e}_1\\
\partial_{1} & \partial_{2} & \partial_{3}\\
u_1 & u_2 & u_3
\end{vmatrix}
\end{equation}

\end{quote}
