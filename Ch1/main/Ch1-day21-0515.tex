\subsubsection{Properties of 2-dim vorticity equation} % H3 title

Consider 2-dim homogeneous incompressible flow in $D$, $\operatorname{div}\textbf{u} = 0$
\begin{equation}
\textbf{u} = (u(x,y,t), v(x,y,t), 0)
\end{equation}
then
\begin{equation}
\frac{D\xi}{Dt} - \left(\zeta\cdot\nabla\right)\textbf{u} + \zeta \cdot \operatorname{div} \textbf{u} = 0
\quad\Rightarrow\quad
\frac{D\xi}{Dt} - \left(\zeta\cdot\nabla\right)\textbf{u} = 0
\end{equation}
where
\begin{equation}
\zeta = \nabla \times \textbf{u} = (0,0,\partial_x v - \partial_y u) = (0,0,\xi)
\end{equation}
now 
\begin{equation}
\left(\xi\cdot\nabla\right)\textbf{u} = \xi\partial_{z}(u,v,0) = 0
\end{equation}
so we have
\begin{equation}
\frac{D\xi}{Dt} = 0
\quad\Leftrightarrow\quad
\frac{\partial \xi}{\partial t} + (\textbf{u}\cdot \nabla)\xi = 0
\end{equation}
Suppose the flow is contained in some region $D$ with a fixed boundary $\partial D$ with boundary condition $\textbf{u}\cdot\textbf{n} = 0$ in $\partial D$.

Now $\operatorname{div} \textbf{u} = 0 \quad\Leftrightarrow\quad u_x + u_y = 0$

Then for $t>0$ $\exists$ a function on $D$, s.t . $u=\partial_y \psi$ and $v = -\partial_x \psi$ then function $\psi$ is the stream function for fixed $t>0$.



streamline lies on level curve of $\psi$​, i.e. 

suppose $(x(s),y(s))$, $s\in I=[a,b]$, is a streamline, so that
\begin{equation}
\frac{d}{ds}\psi(x(s),y(s),0) = \frac{\partial \psi}{\partial x} \frac{dx}{ds} + \frac{\partial \psi}{\partial y} \frac{dy}{ds} = -vu + uv = 0
\end{equation}
then
\begin{equation}
\psi(x(s),y(s),t) = \text{constant}
\end{equation}
on streamline $\forall s$.

Suppose $\textbf{t} = (t_1,t_2)$ is a unit tangent vector on $\partial D$, then $-\textbf{n} = (-t_2,t_1)$ or $\textbf{n}=(t_2,-t_1)$ is the outer normal  on $\partial D$. So, $\textbf{t}\cdot\textbf{n} = 0$ on $\partial D$, that is 
\begin{equation}
\begin{aligned}
& (u,v) \cdot (t_2, -t_1) = 0 \quad\text{on $\partial D$,}\\
\Rightarrow\quad
& (\partial_y \psi, - \partial_x \psi)\cdot (t_2, -t_1) = 0\\
\Rightarrow\quad
& t_2\partial_y \psi +  t_1\partial_x\psi=0\\
\Rightarrow\quad
& \frac{d}{ds}\psi(x(s),y(s),t) = 0\\
\Rightarrow\quad
& \psi(x(s),y(s),t)=C_1 \text{ is a constant on $\partial D$}.
\end{aligned}
\end{equation}
WLOG, May take $\psi(x(s),y(s),t)=0$ on $\partial D$, Also
\begin{equation}
\xi = \partial_x v - \partial_y u = -\partial^2_x\psi-\partial^2_y\psi
= -\nabla^2 \psi= -\Delta \psi
\end{equation}
The equation for $\xi$  for 2-dim, incompressible flow is 
\begin{equation}
\begin{cases}
\displaystyle \frac{D\xi}{Dt} = \frac{\partial \xi}{\partial t} + \left(\textbf{u}\cdot\nabla\right)\xi = 0\\
\displaystyle \nabla^2 \xi = \Delta \xi = 0\\
\displaystyle \psi\bigg|_{\partial D} = 0
\end{cases},\quad\text{with $\textbf{u}=(u,v,0)=(\partial_y\psi, -\partial_x\psi,0)$}
\end{equation}


\begin{quote}
	\textbf{Note :}
\begin{equation}
\begin{aligned}
\left(\textbf{u} \cdot \nabla\right)
&= u\partial_x \xi + v\partial_u \xi\\
&= \partial_y \psi \partial_x\xi + \partial_x \psi \partial_u \xi\\
&= \begin{pmatrix}\partial_x\xi & \partial_y \psi\\ \partial_x \psi & \partial_u \xi\end{pmatrix}\\
&= J(\xi,\psi) =\text{Jocobian of $\xi$ and $\psi$}
\end{aligned}
\end{equation}
we habe
\begin{equation}
\frac{\partial \xi}{\partial t} + J(\xi,\psi) = 0
\end{equation}

\end{quote}
\subsubsubsection{Example} % H4 title

Suppose, at $t=0$, the stream function $\psi(x,y) = \psi(r)$ is a function of radius $r=\sqrt{x^2+y^2}$ with $\partial_r \psi\neq 0$,then
\begin{equation}
\begin{aligned}
u &= \partial_y \psi = \frac{\partial \psi}{\partial r} \frac{\partial r}{\partial y} = \frac{y}{r}\psi_{r}\\
v &= \partial_x \psi = \frac{\partial \psi}{\partial r} \frac{\partial r}{\partial x} = -\frac{x}{r}\psi_{r}
\end{aligned}
\end{equation}
unit tangent vector to cirlce at $(x,y) = (y/r, -x/r)$.

Notice that $\textbf{u}=(u,v)$ is tangent to the circle of radius $r$ with magnitude $\sqrt{u^2+v^2} = |\psi_r|$ and oriented clockwise of $\psi_r>0$ and anti-clockwise if $\psi_r < 0$.

Next, observe that 
\begin{equation}
\xi = -\Delta \psi = \left(\partial^2_{r} \psi + \frac{1}{r}\psi_{r}\right) = \frac{-1}{r}\partial_{r}(r\psi)
\end{equation}
is a function of $r>0$ alone.

Since $\psi_r\neq 0$, by the inverse function theorem, $r$ is a function of $\psi$, Here $\xi = \xi(r)$ is a function $\psi$ 
\begin{equation}
\xi = F(\psi)
\quad\Rightarrow\quad
\xi_{x} = F_{\psi}\psi_{x}, \xi_{y} = F_{\psi}\psi_{y}
\end{equation}
then
\begin{equation}
J(\xi,\psi) = \begin{pmatrix}\partial_x\xi & \partial_y \psi\\ \partial_x \psi & \partial_u \xi\end{pmatrix} =
\begin{pmatrix}
F_{\psi}\xi_x & F_{\psi}\xi_y\\
\partial_x \psi & \partial_u \xi
\end{pmatrix} = 0
\end{equation}
so that 
\begin{equation}
(\textbf{u}\cdot\nabla)\xi = J(\xi,\psi) = 0
\end{equation}


