\subsection{Viscosity, the stress tensor and Navier-Stoke's equation} % H2 title

Consider a surface in a fluid. Suppose the fluid above the surface move with velocity $\textbf{u}_1$ to the right and the fluid below the surface moves with the velocity $\textbf{u}_2$ to the right $\textbf{u}_2$ and $\textbf{u}_1 > \textbf{u}_2$.

Since the molecules of the fluid above $S$ will diffuse to the region below $S$ and riceverse, the fluid above $S$ will slow down the fluid below $S$ will speed up.

We say that the fluid has viscosity and viscosity is the measure of the diffusion of momentum due to the microscopic molecular nature of real fluids. Its effect is to produce a resistance.

This produces shear stress which results from the diffusion of momentum in the fluid.



Consider the rectangular shaped portion of the fluid centered at the point $(x,y,z)$ with side length $(\delta x, \delta y, \delta z)$.

We fixed a time for $t>0$.

Let the component $S_{ij}$ of the stress tensor $S$ be the force per unit area in the $j$-th direction acting across an area element whose mormal is the $i$-th direction.

e.g.

\begin{itemize}
	\item $S_{12}$ is the force per unit area in the $y$-direction acting acorss an area element whose normal in the $x$-direction.
	\item $S_{21}$ is the force per unit area in the $x$-direction acting acorss an area element whose normal in the $y$-direction.

\end{itemize}




We adopt the convention that force acts on the matter on the "minus" side (rear, left, bottom) of the area due to the matter on the "plus" side (front, right, top). By the Newton's third law, forces of equal magnitude and oppsite direction act on the "plus" side dues to the matter on the minus side.

Since the force by the matter in $G_{1}$, on the matter on $G_{2}$ in the $x$-direction across the face $AB$ is $S_{11}\left(x-\delta x/2, y,z\right)\delta y\delta z$.

THerefore the force by the matter in $G$, on the matter in $G_{2}$, in the $x$-direction the face $AB$ and $CD$.
\begin{equation}
S_{11} (x+\delta x/2, y, z)\delta y\delta z - S_{11} (x-\delta x/2, y, z)\delta y\delta z
= \frac{\partial S_{11}}{\partial x}(x^{*},y,z) \delta x\delta y\delta z
\end{equation}
for some $x^{*}\in \left(x-\delta x/2, x+\delta x/2\right)$, so we have
\begin{equation}
\frac{\partial S_{11}}{\partial x}(x^{*},y,z) \delta x\delta y\delta z \approx \frac{\partial S_{11}}{\partial x}(x,y,z) \delta x\delta y\delta z
\end{equation}
Similarly, the sum of forces in other face by the matter $G_2$ on the matter in $G_{1}$ in the $x$-direction is $\approx (\partial S_{21}/\partial y)\delta x\delta y\delta z + (\partial S_{31}/\partial z)\delta x\delta y\delta z$

Therefore the sim of the force in $1$-direction by the matter in $G_2$ on the matter in $G_{1}$ is 
\begin{equation}
\delta F_{1} = \left(\frac{\partial S_{11}}{\partial x} + \frac{\partial S_{21}}{\partial x} + \frac{\partial S_{31}}{\partial x}\right) \delta x \delta y \delta z
\end{equation}
sso the force in $j$-direction by the matter in $G_2$ on the matter $G_1$ is 
\begin{equation}
\delta F_{j} = \left(\frac{\partial S_{1j}}{\partial x} + \frac{\partial S_{2j}}{\partial x} + \frac{\partial S_{3j}}{\partial x}\right) \delta V = (\nabla\cdot S)  \delta V
\end{equation}
where the divergence of stress tensor is 


\begin{equation}
\nabla \cdot S = \nabla \cdot \begin{pmatrix}
S_{1j}\\ S_{2j} \\ S_{3j}
\end{pmatrix}_{j=1,1,3}.
\end{equation}

\subsubsection{Claim} % H3 title

The stress tensor is symmetric, i.e. $S_{ij} = S_{ji}$, $\forall i,j= 1,2,3$.

\subsubsubsection{proof} % H4 title

The force on the volumn element $G_1$, in the $3$-direction ($z$​-comonent) is 
