\begin{quote}
	Test 1
\begin{enumerate}
	\item What is stationart
	\item What is isentropic
	\item What is irrotational

Test 2
	\item What is incompressible

\end{enumerate}
\subsubsection{Definition} % H3 title

\end{quote}
Anirrotational flow is one for which $\xi = \nabla\times\textbf{u} = 0$, which $\textbf{u}$ is the velocity field of the fluid.

\subsubsection{Cor of Berboulli theorem} % H3 title

Suppose the flow is a stationary isentropic irrotational flow with no external forces, then $\displaystyle \frac{1}{2} \lVert \textbf{u}\rVert^2 + w$ is a constant, where $\nabla w = \nabla p/\rho$.

\subsubsubsection{proof} % H4 title

By the proof of the Bernoulli theorem
\begin{equation}
\nabla \left(\frac{1}{2}\lVert \textbf{u}\rVert^2 + w\right) = \textbf{u}\times \left(\nabla\times \textbf{u}\right) = 0
\end{equation}
so that $\displaystyle \frac{1}{2} \lVert \textbf{u}\rVert^2 + w$ is a constant.



\subsubsubsection{Example} % H4 title

Interpretation of vorticity in $2$-dimensional flow. Recall that $\vec{\xi} = \nabla \times \textbf{u}$, for $2$-dimensional flow 
\begin{equation}
\textbf{u}(\textbf{x}) = \textbf{u}(x_1, x_2, t) = (u_1, u_2, 0)
= (u_1(x_1,x_2,t), u_2(x_1,x_2,t), 0)
\end{equation}
we denote
\begin{equation}
\vec{\xi} = (0,0,\xi) =
\begin{vmatrix}
\hat{e}_1 & \hat{e}_2 & \hat{e}_3\\
\partial_1 & \partial_2 & \partial_2\\
u_1 & u_2 & 0
\end{vmatrix}
= (0,\partial_1 u_2- \partial_2 u_1)
\end{equation}
so that $\xi = \partial_1 u_2- \partial_2 u_1$.

If in addition, the fluid is incompressible, then $\operatorname{div}\textbf{u} = \operatorname{div}\left((u_1,u_2,0)\right) = \partial_1u_1 + \partial_2 u_2 = 0$

so that $\partial_1u_1 = - \partial_2 u_2 = \partial_2 (-u_2)$​

Then $\exists$ a function $\psi$, s.t. $u_1 = \partial_2\psi$, $- u_2 = \partial_1\psi$.

Hence
\begin{equation}
\xi = \partial_1 u_2- \partial_2 u_1 = - \partial_1\partial_1\psi - \partial_2\partial_2\psi = \nabla^2 \psi= \Delta \psi
\end{equation}
called Laplacian.

\subsubsubsection{Example} % H4 title

Consider 2 short fluid line elements $\overline{AB}$ and $\overline{AC}$, which are perpendicular at a certain instant, then the $y$-component of velocity at $B$ exceeds that at $A$​ by 
\begin{equation}
u_2(x_1+\delta x_1, x_2, t) - u_2(x_1, x_2, t) \approx \frac{\partial u_2}{\partial x_1} (x_1,x_2,t) \, \delta x.
\end{equation}
Hence $\displaystyle \frac{\partial u_2}{\partial x_1}(x_1,x_2,t)$ represents the instaneous angular velocity of the fluid line element $\overline{AB}$ .

Similarly, $\displaystyle \frac{\partial u_1}{\partial x_2}(x_1,x_2,t)$ represents the instaneous angular velocity (clockwise) of the line element $\overline{AC}$, then $\displaystyle \frac{1}{2}\left(\frac{\partial u_2}{\partial x_1} - \frac{\partial u_1}{\partial x_2}\right) = \frac{1}{2} = \xi$, represent average angular velocity of the two short fluid line elements that happen, at that instant, to be mutually perpendicular.

\begin{quote}
	\textbf{Rmk:}
Vorticity
\end{quote}
