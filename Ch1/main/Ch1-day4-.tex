\subsection{Proof of Euler's Equation} % H2 title

\subsubsection{Balance of Momentum 1 (BM1)} % H3 title

The force per unit area on a point $x\in \partial W$ is $-p\cdot \textbf{n}$, since the total force on $W$ due force the pressure on $\partial W$ is 
\begin{equation}
\begin{aligned}
\textbf{f} &= - \int_{\partial W} p\cdot \textbf{n} dA\\
&= \left(-\int_{\partial W} pn_1 dA, -\int_{\partial W} pn_2 dA, -\int_{\partial W} pn_3 dA\right)\\
&= \left(
-\int_{\partial W} (p,0,0)\cdot \textbf{n} dA,
-\int_{\partial W} (0,p,0)\cdot \textbf{n} dA,
-\int_{\partial W} (0,0,p)\cdot \textbf{n} dA\right)\\
&= \text{by divergence theorem}\\
&=\left(
-\int_{W} \operatorname{div}(p,0,0)dV,
-\int_{W} \operatorname{div}(0,p,0)dV,
-\int_{W} \operatorname{div}(0,0,p)dV\right)\\
&= - \int_{W} (\nabla p) dV
\end{aligned}
\end{equation}


Now, the total force on the fluid due to the pressure 
\begin{equation}
\partial W = -\int_{W} \nabla p dV \label{eq:pf-euler1}
\end{equation}



If $\mathbf{b}(\mathbf{x},t)$ denotes the given body force per unit mass (ex: gravity), then the toal body force is 
\begin{equation}
F_{B} = \int_{W} \rho \cdot \mathbf{b}\cdot dV.\label{eq:pf-euler2}
\end{equation}
By equations ($\ref{eq:pf-euler1}$) and ($\ref{eq:pf-euler2}$), the force per unit volume is $-\nabla p + \rho \textbf{b}$

By the Newton's second law, 
\begin{equation}
\frac{D}{Dt} (\delta m\, \mathbf{u})
= \frac{D}{Dt} (\delta m\, \mathbf{u}(\mathbf{x},t))
= (-p + \rho \textbf{b})\delta V
\end{equation}
we then have
\begin{equation}
\Rightarrow \rho \frac{D\mathbf{u}}{Dt} = - \nabla p + \rho \textbf{b} \quad \text{BM1 (Euler equation)}
\end{equation}

\subsubsection{Balance of Momentum 2 (BM2)} % H3 title

WLOG. we write $\textbf{u}$ for $\textbf{u} = (u_1, u_2, u_3)$ and $b$ for $\textbf{b}$, Integral from of balance of momentum

By (BM1) 
\begin{equation}
\rho \frac{\partial \textbf{u}}{\partial t} = - \rho (\textbf{u}\cdot \nabla) \textbf{u} - \nabla p + \rho \textbf{b}\label{eq:pf-euler4}
\end{equation}
then by $\ref{eq:pf-euler4}$  and continuity equation
\begin{equation}
\frac{\partial}{\partial t} (\rho \textbf{u}) = \rho_{t} \textbf{u} + \rho \textbf{u}_t = -\operatorname{div}(\rho\textbf{u}) \textbf{u} - \rho (\textbf{u}\cdot \nabla )\textbf{u} - \nabla p + \rho \textbf{b}
\end{equation}
Let $\textbf{e}$ be a fixed vector in space, then
\begin{equation}
\begin{aligned}
\textbf{e} \cdot \frac{\partial }{\partial t} (\rho \textbf{u})
&= - \operatorname{div}(\rho\textbf{u}) \textbf{u} \cdot \textbf{e}
- \rho (\textbf{u}\cdot \nabla )\textbf{u}\cdot \textbf{e}
- (\nabla p)\cdot \textbf{e} + \rho \textbf{b}\cdot \textbf{e}\\
&= - \operatorname{div}\left(\rho \textbf{u}(\textbf{u}\cdot\textbf{e})\right)
- \operatorname{div}\left(p\textbf{e}\right)
+ \rho \textbf{b}\cdot\textbf{e}\\
&= - \operatorname{div}\left(\rho \textbf{u}(\textbf{u}\cdot\textbf{e}) + p\textbf{e}\right)
+ \rho \textbf{b}\cdot\textbf{e}
\end{aligned}\label{eq:pf-euler5}
\end{equation}
Since, we have 

\begin{enumerate}
	\item the divergence of $p\textbf{e}$

\begin{equation}
\begin{aligned}
\operatorname{div}(p \textbf{e})
&= \operatorname{div}\left((pe_1,pe_2,pe_3)\right)\\
&= \frac{\partial p}{\partial x_1} e_1 + \frac{\partial p}{\partial x_2} e_2 + \frac{\partial p}{\partial x_3} e_3\\
&= \sum_{i=1}^{3} \frac{\partial p}{\partial x_i} e_i = \nabla p \cdot \textbf{e}
\end{aligned}
\end{equation}

\end{enumerate}
\begin{enumerate}
	\item the divergence of $\rho \textbf{u}(\textbf{u}\cdot\textbf{e})$

\begin{equation}
\begin{aligned}
\operatorname{div}(\rho \textbf{u}(\textbf{u}\cdot \textbf{e}))
&= \sum_{i=0}^{3} \frac{\partial}{\partial x_i}\left[\rho u_i (\textbf{u}\cdot \textbf{e})\right]\\
&= \sum_{i=0}^{3}\left(\frac{\partial }{\partial x_i}(\rho u_i)\right)(\textbf{u}\cdot \textbf{e})
+  \sum_{i=0}^{3}(\rho u_i)\left(\frac{\partial }{\partial x_i} (\textbf{u}\cdot \textbf{e})\right)\\
&= \operatorname{div}(\rho \textbf{u})(\textbf{u}\cdot \textbf{e})
+ \rho (\textbf{u}\cdot \nabla) (\textbf{u}\cdot \textbf{e})
\end{aligned}
\end{equation}

\end{enumerate}
Hence, if $W$ is a fixed region in space in the fluid 
\begin{equation}
\begin{aligned}
\textbf{e}\cdot \frac{d}{dt} \int_{W} \rho \textbf{u} dV
&= \int _{W} e \cdot \frac{d}{dt}(\rho \textbf{u}) dV \\
&= -\int_{W} \operatorname{div}(p\textbf{e} + \rho \textbf{u}(\textbf{u}\cdot \textbf{e})) dV
+ \int_{W} \rho \textbf{b}\cdot \textbf{e} dV\\
&= \text{By divergence theorem.}\\
&= -\int_{\partial W} (p\textbf{e} + \rho \textbf{u} (\textbf{u}\cdot \textbf{e})) \cdot \textbf{n}dA
+ \int_{W} \rho \textbf{b}\cdot \textbf{e} dV\\
&= -\int_{\partial W} p\textbf{e} \cdot \textbf{n}dA
-\int_{\partial W}\rho \textbf{u} (\textbf{u}\cdot \textbf{e})\cdot \textbf{n}dA
+ \int_{W} \rho \textbf{b}\cdot \textbf{e} dV,\quad \forall \textbf{e}\in \mathbb{R}^{n},n=2\text{ or }3
\end{aligned}
\end{equation}
then 
\begin{equation}
\frac{d}{dt} \int_{W} \textbf{u} dV = -\int_{\partial W} p\textbf{n}dA
- \int_{\partial W} \rho(\textbf{u}\cdot\textbf{n}) \textbf{u}dA + \int_{\partial W} \rho \textbf{b}\cdot \textbf{e} dV
\end{equation}
or 
\begin{equation}
\frac{d}{dt} \int_{W} \textbf{u} dV = -\int_{\partial W} \left(p\textbf{n} + \rho(\textbf{u}\cdot\textbf{n}) \textbf{u}\right)dA + \int_{\partial W} \rho \textbf{b}\cdot \textbf{e} dV,\quad \text{BM2}
\end{equation}
and BM2 is also the Integral form of balance of momentum.

\begin{quote}
	Note: The quantity $p\textbf{n} + \rho \textbf{u}(\textbf{u}\cdot \textbf{n})$ is the momentum per unit area crossing $\partial W$ when $\textbf{n}$ is unit vector outer normal to $\partial W$.
\end{quote}

\subsubsection{Balance of Momentum 3 (BM3)} % H3 title

Let $D$ be a region that the fluid is moving and $x\in D$

Let $\varphi (\textbf{x},t)$ be the trajectory of the partiacle that is at point $x$, i.e. $\varphi$ satisfies 
\begin{equation}
\begin{aligned}
\frac{\partial}{\partial t}\varphi(\textbf{x},t) &= \textbf{u} \left(\varphi(\textbf{x},t),t\right)
&&\text{$\forall t > 0$ at time $t$}
\\
\varphi(\textbf{x},0) &= x && \varphi(\textbf{x},t) = \varphi_t(\textbf{x})
\end{aligned}
\end{equation}
We will assume that $\varphi$ is smooth and for fixed $t$, $\varphi_t:t\to\varphi(\textbf{x},t)$ is invertible.

\begin{quote}
	$\varphi_t$ doesn't mean $\partial/\partial t$ here!
\end{quote}

We called $\varphi$ is the fluid flow map.



If $W$ is the a region in $D$, then $W_t := \varphi_{t}(W)$ is the region of the fluid at time $t$ whose initial position is in $W$ at time $t$.

Then by the balance of momentum
\begin{equation}
\frac{d}{dt} \int_{W_t} \rho \textbf{u} dV = \textbf{F}_{\partial W_t} + \int_{W_t} \rho \textbf{b}dV,
\end{equation}
where $\textbf{F}_{\partial W_t}$ is the force on $\partial W_t$ due to perssure, i.e.
\begin{equation}
\textbf{F}_{\partial W_t} = - \int_{\partial W_t} p\textbf{n}dA = -\int_{W_t} \nabla pdV
\end{equation}
so that 
\begin{equation}
\frac{d}{dt} \int_{W_t} \rho \textbf{u} dV = -\int_{W_t} \nabla pdV + \int_{W_t} \rho \textbf{b}dV,\quad \text{BM3}
\end{equation}


\begin{quote}
	Recall
\begin{equation}
\frac{d}{dt}(\delta V) = (\operatorname{div} \textbf{u}) \delta V
\end{equation}
for an infitesmal volume $\delta V$ of fluid moving in te fluid with the fluid velocity, We will now given a rigorous proof of the result.
\end{quote}

Now that 
\begin{equation}
\begin{aligned}
\operatorname{volume}(W_t)
&= \int_{W_t} 1 dV\\
&= \int_{W_t} 1 dy,\quad \text{put $y$ to be $\varphi_t(\textbf{x})$}\\
&= \int_{W} J(\textbf{x},t) d\textbf{x},
\end{aligned}
\end{equation}
where $J(\textbf{x},t)$ is the Jocobian determinant of the map $\varphi_t$ , so that 
\begin{equation}
\frac{d}{dt}\operatorname{volume}(W_t) = \int_{W} \frac{\partial}{\partial t} J(\textbf{x},t)d\textbf{x}
\end{equation}

