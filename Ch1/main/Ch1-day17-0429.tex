\subsubsubsection{Example} % H4 title

Consider the line votex in cylinderical polar coordinate $(r,\theta,z)$ by
\begin{equation}
\textbf{u}\left(r,\theta,z\right) = \left(0,\frac{k}{r},0\right) = \frac{k}{r} \textbf{e}_{\theta}
\end{equation}
where $\textbf{u}$ is the velocity vector field field and the basis 
\begin{equation}
\textbf{e}_r = \left(1,0,0\right),\quad
\textbf{e}_\theta = \left(0,1,0\right),\quad
\textbf{e}_z = \left(0,0,1\right).
\end{equation}
The vorticity in cylindrical coordinate given 
\begin{equation}
\nabla \times \textbf{u} = \frac{1}{r}\begin{vmatrix}
\textbf{e}_r & \textbf{e}_r & \textbf{e}_z \\
\partial_r & \partial_\theta & \partial_z\\
u_r & u_\theta & u_z
\end{vmatrix}
= \begin{vmatrix}
\textbf{e}_r & \textbf{e}_r & \textbf{e}_z \\
\partial_r & \partial_\theta & \partial_z\\
0 & k & 0
\end{vmatrix} = 0
\end{equation}
Hence, the vorticity is zero $\forall r\neq 0$, thus although the fluid is rotating in global sence, the flow is irrotational. 

\begin{quote}
	Note that consider $z$ momentarily $\perp$ fluid line elements $\overline{AB}$  and $\overline{AC}$ at $\theta=0$.
$\overline{AC}$ is rotating in an anticlockwise sense, it will continue to line along the circular streamline as time preceeds.
\end{quote}

On the other hands, since $u_\theta$ is a decreasing function of $r>0$, $\overline{AB}$ is rotating clockwide so that $\overline{AB}$ has the equal and 

\subsubsection{The Rankine vortex} % H3 title

Consider a axisymmetric flow $\textbf{u} = (u_1,u_2,0)$ with vorticity 
\begin{equation}
\xi = \begin{cases}
\Omega & \text{for } r\leq a\\
0 & \text{for } r> a\\
\end{cases},\quad r = \sqrt{x^2 + y^2}
\end{equation}
which is incompressible. By the previous result $\exists$ a function s.t. $u=\psi_y$, $v=\varphi_x$ and 
\begin{equation}
-\nabla \psi = \xi
= \begin{cases}
\Omega & \text{for } r\leq a\\
0 & \text{for } r>a
\end{cases}
\end{equation}
Assume $\psi = \psi (r)$, Thus, 
\begin{equation}
\nabla \psi = \frac{\partial^2 \psi(r)}{\partial r^2} + \frac{1}{r} \frac{\partial \psi}{\partial r}
= \frac{1}{r}\frac{\partial}{\partial r}\left(r\frac{\partial \psi}{\partial r}\right)
\end{equation}
For $r\leq a$, 
\begin{equation}
\xi = \frac{1}{r}\frac{\partial}{\partial r}\left(r\frac{\partial \psi}{\partial r}\right) = -\Omega
\quad\Rightarrow\quad
\frac{\partial}{\partial r}\left(r\frac{\partial \psi}{\partial r}\right) = -r\Omega
\quad\Rightarrow\quad
r \frac{\partial \psi}{\partial r} = -\frac{\Omega r^2}{2} + C_1
\end{equation}
where $C_1$ is a constant, last we have
\begin{equation}
\frac{\partial \psi}{\partial r} = - \frac{\Omega r}{2} + \frac{C_1}{r}
\end{equation}
Suppose $\psi'(r)$ is bounded at $r=0$, so that $C_1 = 0$.
\begin{equation}
\psi(r) = -\frac{\Omega r^2}{4},\quad r\leq a
\end{equation}
For $r>a$,
\begin{equation}
\xi = \frac{1}{r}\frac{\partial}{\partial r}\left(r\frac{\partial \psi}{\partial r}\right) = 0
\quad\Rightarrow\quad
\frac{\partial}{\partial r}\left(r\frac{\partial \psi}{\partial r}\right) = 0
\quad\Rightarrow\quad
r \frac{\partial \psi}{\partial r} = C_2
\end{equation}
then
\begin{equation}
\frac{\partial \psi}{\partial r} = \frac{C_2}{r}
\quad\Rightarrow\quad
\psi = C_2 \ln r + C_3
\end{equation}
Suppose $\psi'(r)$ is bounded at $r\to \infty$, so that $C_2 = 0$, and $\psi(a) = \psi(a^{+}) = \psi(a^{-})$ so that 
\begin{equation}
\psi(r) = -\frac{\Omega a^2}{4} ,\quad r>a
\end{equation}
In conclude
\begin{equation}
\psi(r) =
\begin{cases}
\displaystyle -\frac{\Omega r^2}{4}& r\leq a\\
\displaystyle -\frac{\Omega a^2}{4}& r > a
\end{cases}
\end{equation}



