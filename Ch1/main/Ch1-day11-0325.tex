\subsubsection{Bernoullis' Theorem} % H3 title

For stationary. isentropic flows and in the absence of external forces,
\begin{equation}
\textbf{u} \cdot \nabla \left(w + \frac{1}{2}\lVert\textbf{u}\rVert^2\right) = 0
\Leftrightarrow
\frac{d}{dt}\left(\frac{1}{2}\lVert\textbf{u}\rVert^2+w\right)(\textbf{x}(t)) = 0
\label{eq: bernoulli }
\end{equation}
which means $\displaystyle \frac{1}{2}\lVert\textbf{u}\rVert^2 + w$ is constant along stream, where $x(t)$ satisfies $\displaystyle \frac{d}{dt}\textbf{x}(t) = \textbf{u}(\textbf{x}(t))$ (Streamline of flow).

\begin{quote}
	The same result holds if the force $\textbf{b}$ is conservative, i.e. $\textbf{b}=-\nabla \varphi$ for some function $\varphi, w$ replaced by $w+\varphi$ in above statement $(\ref{eq: bernoulli })$.
\end{quote}

\begin{quote}
	Rmk:
Note, if flow is stationary, homogeneous (i.e. $\rho = \rho_0$ is constant in $D$)  and incompressible, the flow is isentropic with $w=p/\rho_0$ hence the statement $(\ref{eq: bernoulli })$ will holds.
\end{quote}

\subsubsubsection{Proof} % H4 title

By Lemma 1,
\begin{equation}
\frac{1}{2}\nabla\lVert\textbf{u}\rVert^2 = \left(\textbf{u}\cdot \nabla\right)\textbf{u} + \textbf{u}\times\left(\nabla\times\textbf{u}\right).
\end{equation}
Since the flow is steady, we have
\begin{equation}
\textbf{u}_t + \left(\textbf{u}\cdot \nabla\right) \textbf{u} = -\nabla w
\end{equation}
so that 
\begin{equation}
\nabla w + \left(\textbf{u}\cdot \nabla\right) \textbf{u} = -\textbf{u}_t = 0
\quad\Rightarrow\quad
\left(\textbf{u}\cdot \nabla\right) \textbf{u} = - \nabla w.
\label{eq: steady flow result }
\end{equation}
Consider, 
\begin{equation}
\nabla\left(\frac{1}{2}\lVert\textbf{u}\rVert^2 + w\right) = \left(\textbf{u}\cdot \nabla\right)\textbf{u} + \textbf{u}\times\left(\nabla\times\textbf{u}\right) + \nabla w
\end{equation}
using the result of  $(\ref{eq: steady flow result })$, we have 
\begin{equation}
\nabla\left(\frac{1}{2}\lVert\textbf{u}\rVert^2 + w\right) = \textbf{u}\times\left(\nabla\times\textbf{u}\right)
\end{equation}
so that 
\begin{equation}
\textbf{u}\cdot\nabla\left(\frac{1}{2}\lVert\textbf{u}\rVert^2 + w\right) = \textbf{u}\cdot\left(\textbf{u}\times\left(\nabla\times\textbf{u}\right)\right) = 0
\end{equation}
Taking the divergence, 
\begin{equation}
\nabla w = -\operatorname{div}\left((\textbf{u}\cdot\nabla)\textbf{u}\right),\quad \text{in $D$.}
\end{equation}
By $(\ref{eq: steady flow result })$, $\displaystyle \frac{\partial W}{\partial \textbf{n}}\Bigg|_{\partial D} = \nabla w\cdot \textbf{n} = - (\textbf{u}\cdot\nabla)\textbf{u}\cdot \textbf{n}$, since $w$ satisfies the elliptic PDE with boundary condition, it shows that $w$ is a finction independent of the time $t$:
\begin{equation}
\frac{d}{dt}\left(\frac{1}{2}\lVert\textbf{u}\rVert^2+w\right)(\textbf{x}(t))
=\textbf{u} \cdot \nabla \left(w + \frac{1}{2}\lVert\textbf{u}\rVert^2\right)(\textbf{x}(t))
= 0
\end{equation}
or 
\begin{equation}
\frac{d}{dt}\left(\frac{1}{2}\lVert\textbf{u}\rVert^2+w\right)(\textbf{x}(2)) - \frac{d}{dt}\left(\frac{1}{2}\lVert\textbf{u}\rVert^2+w\right)(\textbf{x}(1)) = \int_{t_1}^{t_2} \left(\frac{1}{2}\lVert\textbf{u}\rVert^2+w\right)dt = 0
\end{equation}
 That is $\displaystyle \frac{1}{2}\lVert\textbf{u}\rVert^2+w$ is a constant along stream line.

\subsubsubsection{Example} % H4 title

Consider a fluid flow in a channel

\underline{FIg}

Suppose the pressure is a function of $\textbf{x}$ only and the pressure $p_1$ at $\textbf{x}=0$ is greater than the pressure $p_2$. Then the fluid will flow from left to right with the velocity $\textbf{u}(x,y,t) = (u_1(x,y),0,0)$ and the pressure $\textbf{p}(x,y,t) = p(\textbf{x})$. Suppose the density of fluid $\rho=\rho_0$ is a constant  and the fluid is incompressible.






