Now, we first calculate
\begin{equation}
\begin{aligned}
\operatorname{div}\left((\textbf{u}\cdot \nabla)\, \textbf{u}\right)
&= \sum_{i} \frac{\partial}{\partial x_i}\left(\textbf{u}\cdot \nabla\right) u_i\\
&= \sum_{i} \frac{\partial}{\partial x_i}\left(\sum_{j}u_j\frac{\partial}{\partial x_j}\right) u_i\\
&= \sum_{i}\sum_{j}\frac{\partial}{\partial x_i}\left(\left(u_j\frac{\partial}{\partial x_j}\right) u_i\right)\\
&= \sum_{i}\sum_{j}
\left(\frac{\partial}{\partial x_i}\left(u_j\frac{\partial}{\partial x_j}\right)\right) u_i
+\left(u_j\frac{\partial}{\partial x_j}\right)\left(\frac{\partial}{\partial x_i} u_i\right)\\
&= \sum_{i}\sum_{j}
\left(\left(\frac{\partial u_j}{\partial x_i}\right)\left(\frac{\partial}{\partial x_j}\right)\right) u_i
+\left(u_j\frac{\partial}{\partial x_i}\frac{\partial}{\partial x_j}\right) u_i
+\left(u_j\frac{\partial}{\partial x_j}\right)\left(\frac{\partial}{\partial x_i} u_i\right)\\
&= \sum_{i}\sum_{j}
\left(\frac{\partial u_j}{\partial x_i}\right)\left(\frac{\partial u_i}{\partial x_j}\right)
+2\left(u_j\frac{\partial}{\partial x_j}\right)\left(\frac{\partial u_i}{\partial x_i}\right)\\
&= \sum_{i}\sum_{j}
\left(\frac{\partial u_j}{\partial x_i}\right)\left(\frac{\partial u_i}{\partial x_j}\right)
+2\sum_{i}\sum_{j}\left(u_j\frac{\partial}{\partial x_j}\right)\left(\frac{\partial u_i}{\partial x_i}\right)\\
&= \sum_{i}\sum_{j}
\left(\frac{\partial u_j}{\partial x_i}\right)\left(\frac{\partial u_i}{\partial x_j}\right)
+2\sum_{j} \left(u_j\frac{\partial}{\partial x_j}\right)\operatorname{div}\textbf{u}\\
&= \sum_{i}\sum_{j}
\left(\frac{\partial u_j}{\partial x_i}\right)\left(\frac{\partial u_i}{\partial x_j}\right)
+2\left(\textbf{u}\cdot \nabla\right)\operatorname{div}\textbf{u}\\
\end{aligned}
\end{equation}


notice $\operatorname{div}\textbf{u} = 0$, we then have
\begin{equation}
\operatorname{div}\left((\textbf{u}\cdot \nabla)\, \textbf{u}\right) + \frac{\nabla p}{\rho}
= \sum_{i,j}\left(\frac{\partial u_j}{\partial x_i}\right)\left(\frac{\partial u_i}{\partial x_j}\right) + \frac{\nabla p}{\rho} = 0
\end{equation}
So the Euler's equation becomes
\begin{equation}
\nabla p = -\rho\sum_{i,j}\left(\frac{\partial u_j}{\partial x_i}\right)\left(\frac{\partial u_i}{\partial x_j}\right).
\end{equation}
Since the fluid is confined in fixed region of space by $D$, therefore the fluid cannot move into $\mathbb{R}^3\setminus D$ (out side), so the normal component of the fluid satisfied 
\begin{equation}
\textbf{u}\cdot\textbf{n}\Bigg|_{\partial D} = 0
\quad\Rightarrow\quad
\frac{\partial \textbf{u}}{\partial t}\cdot \textbf{n}\Bigg|_{\partial D} = 0,
\end{equation}
then by Euler's equation, we have
\begin{equation}
\frac{\partial p}{\partial \textbf{n}}\Bigg|_{\partial D} = \nabla p \cdot \textbf{n} = -\rho (\textbf{u}\cdot\nabla)\textbf{u}\cdot \textbf{n}\Bigg|_{\partial D}.
\end{equation}
For the flat boundary surfaces, e.g. the wall of flixed rectangular box, $\textbf{n} = (n_1,n_2,n_3)$ and 
\begin{equation}
\begin{aligned}
\textbf{n}\cdot \left(\left(\textbf{u}\cdot \nabla\right)\textbf{u}\right)
&= \sum_{i,j} n_i \left(u_i\frac{\partial}{\partial x_j} u_i\right)\\
&= \sum_{i,j} u_j \frac{\partial}{\partial x_j}(u_in_i)\\
&= \sum_{j} u_j \frac{\partial}{\partial x_j}(\textbf{u}\cdot\textbf{n}) = 0,\quad \text{Since $\textbf{u}\cdot\textbf{n}\Bigg|_{\partial D} = 0$}
\end{aligned}
\end{equation}
on each flat surface. Since 
\begin{equation}
\frac{\partial p}{\partial n}\Bigg|_{\partial D} = 0
\end{equation}
on each flat surface. Then one can solve 
\begin{equation}
\begin{cases}
\displaystyle \nabla p = - \rho \sum_{i,j}\frac{\partial u_j}{\partial x_i}\frac{\partial u_i}{\partial x_j},\\
\displaystyle  \frac{\partial p}{\partial \textbf{n}}\Bigg|_{\partial D} = 0,\quad \text{(If the surface is flat.)}
\end{cases}
\end{equation}
using standard PDE method.

\begin{quote}
	Rmk:
If the surface are not flat, the equations become
\begin{equation}
\begin{cases}
\displaystyle \nabla p
= - \rho \sum_{i,j}\frac{\partial u_j}{\partial x_i}\frac{\partial u_i}{\partial x_j},\\
\displaystyle \frac{\partial p}{\partial \textbf{n}}\Bigg|_{\partial D}
= - \left(\textbf{u}\cdot\nabla\right)\textbf{u}\cdot \textbf{n}\Bigg|_{\partial D}
\end{cases}
\end{equation}

\end{quote}


\subsection{Consevation of Energy} % H2 title

For fluid moving in a domain $D$ with velocity $\textbf{u}$, the kinetic energy of the fluid in $W$ is 
\begin{equation}
E_{\rm kinetic} = \frac{1}{2}\int_{W}\rho \cdot \lVert\textbf{u}\rVert^2 dV,
\end{equation}
where $\textbf{u} = (u_1,u_2,u_3)$ and $\lVert\textbf{u}\rVert^2 = \sqrt{u_1^2+u_2^2+u_3^2}$.





\begin{quote}
	Rmk:
When $W = D$ and the fluid is incompressible with no external force acting on fluid, we have
\begin{equation}
\begin{aligned}
\frac{d}{dt}E_{\rm kinetic}
&= \frac{1}{2}\frac{d}{dt}\int_{W}\rho \cdot \lVert\textbf{u}\rVert^2 dV\\
&= \frac{1}{2}\int_{D}\frac{\partial}{\partial t}\left(\rho \lVert\textbf{u}\rVert^2\right)dV\\
&= \frac{1}{2}\int_{D}\rho_t \lVert\textbf{u}\rVert^2 + \rho \frac{\partial}{\partial t} \lVert\textbf{u}\rVert^2 dV\\
&= \frac{1}{2}\int_{D} \bigg(
-\operatorname{div}(\rho\textbf{u})\lVert\textbf{u}\rVert^2
+2\rho \textbf{u}\cdot \textbf{u}_t
\bigg)dV\\
&= \frac{1}{2}\int_{D} \bigg(
-\operatorname{div}\left(\rho\textbf{u}\lVert\textbf{u}\rVert^2\right)
+(\rho\textbf{u})\nabla\lVert\textbf{u}\rVert^2
+2\rho \textbf{u}\cdot \textbf{u}_t
\bigg)dV
\end{aligned}
\end{equation}
also notice that the euler equation $\textbf{u}_t+(\textbf{u}\cdot \nabla)\textbf{u}+\nabla p/\rho=0$ , pluin
\begin{equation}
\begin{aligned}
\frac{d}{dt}E_{\rm kinetic}
&= \frac{1}{2}\int_{D} \bigg(
-\operatorname{div}\left(\rho\textbf{u}\lVert\textbf{u}\rVert^2\right)
+(\rho\textbf{u})\cdot\nabla\lVert\textbf{u}\rVert^2
+2\rho \textbf{u}\cdot \textbf{u}_t
\bigg)dV\\
&= \frac{1}{2}\int_{D} \bigg(
-\operatorname{div}\left(\rho\textbf{u}\lVert\textbf{u}\rVert^2\right)
+(\rho\textbf{u})\cdot 2\left((\textbf{u}\cdot \nabla) \textbf{u}\right)
+2\rho \textbf{u}\cdot\left(-(\textbf{u}\cdot \nabla)\textbf{u}-\frac{\nabla p}{\rho}\right)
\bigg)dV\\
&= \frac{1}{2}\int_{D} \bigg(
-\operatorname{div}\left(\rho\textbf{u}\lVert\textbf{u}\rVert^2\right)
+2\rho\textbf{u}\cdot(\textbf{u}\cdot \nabla) \textbf{u}
-2\rho \textbf{u}\cdot(\textbf{u}\cdot \nabla)\textbf{u}
-2\rho \textbf{u}\cdot\frac{\nabla p}{\rho}
\bigg)dV\\
&= \frac{1}{2}\int_{D} \bigg(
-\operatorname{div}\left(\rho\textbf{u}\lVert\textbf{u}\rVert^2\right)
-2\rho \textbf{u}\cdot\frac{\nabla p}{\rho}
\bigg)dV\\
&= -\frac{1}{2}\int_{D}
\operatorname{div}\left(\rho\textbf{u}\lVert\textbf{u}\rVert^2\right)dV
-\int_{D}\left(\textbf{u}\cdot\nabla p\right)dV\\
&= -\frac{1}{2}\int_{\partial D}
\rho\lVert\textbf{u}\rVert^2\textbf{u}\cdot\textbf{n}d\sigma
-\int_{D}\left(\textbf{u}\cdot\nabla p\right)dV\\
\end{aligned}
\end{equation}
Since, $\textbf{u}\cdot\textbf{n} = 0$,
\end{quote}



\begin{quote}
	Rmk:
When $E_{\rm kinetic}$
\end{quote}

