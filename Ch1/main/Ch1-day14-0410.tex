\begin{quote}
	Recall, \textbf{Proposition}:
For $\textbf{y} = \textbf{x} + \textbf{h}$
\begin{equation}
\textbf{u}(\textbf{y},t) = \textbf{u}(\textbf{x},t) + \textbf{D}(\textbf{x},t)\cdot \textbf{h} + \frac{1}{2}\mathbf{\xi}(\textbf{x},t)\cdot \textbf{h} + \mathcal{O}\left(\lVert \textbf{h}\rVert^2\right),
\end{equation}
$\textbf{D}$ is symmetric, $D$ is diagonalizable. That is $\exists$ an orthogonal matrix $\textbf{U}$, s.t.
\begin{equation}
\widetilde{\textbf{D}} = \textbf{U} \textbf{D} \textbf{U}^{T} = \begin{pmatrix}
d_1 & 0 & 0\\
0 & d_2 & 0\\
0 & 0 & d_3
\end{pmatrix}
\end{equation}
where $\textbf{U}\textbf{U}^{T} = \textbf{U}^{T}\textbf{U} = \textbf{I}$, for some constants $d_1, d_2, d_3$.
Hence $\exists$ a rotation of the coordinate system about $\textbf{x}$ in the new coordinate system
\begin{equation}
\textbf{y} \to \widetilde{\textbf{y}} = \textbf{x} + \widetilde{\textbf{h}},\quad
\textbf{D}\to \widetilde{\textbf{D}}
\end{equation}
Keeping $\textbf{x}$ fixed (stationary fluid) and assuming the fluid is $\textbf{u}(\textbf{x},t) = \textbf{u}(\textbf{x})$, $\forall t$.
\end{quote}

\subsubsubsection{Case 1} % H4 title

If we ignore all terms in Proposition except $\textbf{D}\cdot \textbf{h}$, we get the motion of the fluid is 
\begin{equation}
\frac{d\textbf{y}}{dt} = \textbf{u}(\textbf{y})
\quad\Rightarrow\quad
\frac{d\textbf{y}}{dt} = \textbf{D}\cdot\textbf{h}
\quad\Rightarrow\quad
\frac{d\textbf{x}}{dt} + \frac{d\textbf{h}}{dt}= \textbf{0} + \frac{d\textbf{h}}{dt} = \textbf{D}\cdot\textbf{h}
\end{equation}
that is 
\begin{equation}
\frac{d\textbf{h}}{dt} = \textbf{D}\cdot\textbf{h}
\end{equation}
or is we using the new coordinate system, one has
\begin{equation}
\frac{d\widetilde{h}_i}{dt} = d_i\widetilde{h}_i,\quad \forall i=1,2,3
\end{equation}
Hence the vector field $\textbf{D}\cdot\textbf{h}$ is the thus merely expanding or contracting along the new directions $\widetilde{\textbf{e}}_1$, $\widetilde{\textbf{e}}_2$ and $\widetilde{\textbf{e}}_3$. The rate of change of the volumn of a box with side length $\widetilde{h}_1$, $\widetilde{h}_2$ and $\widetilde{h}_3$ is 
\begin{equation}
\begin{aligned}
\frac{dV}{dt}
&= \frac{d}{dt}\left(\widetilde{h}_1\widetilde{h}_2\widetilde{h}_3\right) \\
&= \widetilde{h}_2\widetilde{h}_3\frac{d}{dt}\left(\widetilde{h}_1\right)
+ \widetilde{h}_1\widetilde{h}_3\frac{d}{dt}\left(\widetilde{h}_2\right)
+ \widetilde{h}_1\widetilde{h}_2\frac{d}{dt}\left(\widetilde{h}_3\right)\\
&= \widetilde{h}_2\widetilde{h}_3\widetilde{h}_1 \widetilde{d}_1
+ \widetilde{h}_1\widetilde{h}_3\widetilde{h}_2 \widetilde{d}_2
+ \widetilde{h}_1\widetilde{h}_2\widetilde{h}_3 \widetilde{d}_3\\
&= \widetilde{h}_1\widetilde{h}_2\widetilde{h}_3 \left(\widetilde{d}_1+\widetilde{d}_2+\widetilde{d}_3\right)
\end{aligned}
\end{equation}
and notice that 
\begin{equation}
\begin{aligned}
\left(\widetilde{d}_1+\widetilde{d}_2+\widetilde{d}_3\right)
&= \Tr\left(\widetilde{\textbf{D}}\right)
=\Tr\left(\textbf{U}\textbf{D}\textbf{U}^{T}\right)\\
&= \Tr\left(\textbf{U}\right)\Tr\left(\textbf{D}\right)\Tr\left(\textbf{U}^{T}\right)\\
&= \Tr\left(\textbf{U}\right)\Tr\left(\textbf{U}^{T}\right)\Tr\left(\textbf{D}\right)\\
&= \Tr\left(\textbf{U}\textbf{U}^{T}\right)\Tr\left(\textbf{D}\right)
= \Tr\left(\textbf{D}\right)\\
&= \Tr\left(\frac{1}{2}\left((\nabla \textbf{u}) + (\nabla \textbf{u})^{T}\right)\right)\\
&= \operatorname{div}\textbf{u}
\end{aligned}
\end{equation}
so that the volumn elements change at a rate $\widetilde{h}_1\widetilde{h}_2\widetilde{h}_3 \cdot(\operatorname{div}\textbf{u})$

\subsubsubsection{Case 2} % H4 title

If the term $\displaystyle \frac{1}{2}\xi(\textbf{x})\times\textbf{h}$ dominate 
\begin{equation}
\frac{d\textbf{h}}{dt} = \eta\times\textbf{h},
\quad\text{where }
\eta = \eta(\textbf{x}) = \frac{\xi(\textbf{x})}{\lVert\xi(\textbf{x})\rVert}
\end{equation}
Let $\textbf{R}_0$ be a rotation which take $\eta$ to $(0,0,1)$ and
\begin{equation}
\begin{aligned}
\textbf{h}_0 &= (h_{0,1}, h_{0,2}, h_{0,3}) = (h_{1}(0), h_{2}(0), h_{3}(0))\\
\widetilde{\textbf{h}}_0 &= (\widetilde{h}_{0,1}, \widetilde{h}_{0,2}, \widetilde{h}_{0,3}) = (\widetilde{h}_{1}(0), \widetilde{h}_{2}(0), \widetilde{h}_{3}(0))\\
\end{aligned}
\end{equation}
where $\textbf{R}_0 = \widetilde{\textbf{h}}_0$ and $\textbf{R}_0(\eta) = (0,0,1)$. 
\begin{equation}
\textbf{R}_0(\textbf{h}(t)) = \widetilde{\textbf{h}}(t) = \left(\widetilde{h}_1, \widetilde{h}_2, \widetilde{h}_3\right)
\end{equation}
then
\begin{equation}
\begin{aligned}
\textbf{R}_0\left(\frac{d\textbf{h}}{dt}\right)
&= \textbf{R}_0(\eta\times\textbf{h})\\
&=  \textbf{R}_0(\eta) \times \textbf{R}_0(\textbf{h})\\
&= (0,0,1)\times\widetilde{\textbf{h}}\\
&= (-\widetilde{h}_2, \widetilde{h}_1, 0)
\end{aligned}
\end{equation}
also
\begin{equation}
\textbf{R}_0\left(\frac{d\textbf{h}}{dt}\right)  = \frac{d\textbf{R}_0(\textbf{h})}{dt}
= \frac{d\widetilde{\textbf{h}}}{dt} = (\widetilde{h}_1, \widetilde{h}_2, \widetilde{h}_3)
\end{equation}
Then solving the ODE system
\begin{equation}
\frac{d}{dt}(\widetilde{h}_1, \widetilde{h}_2, \widetilde{h}_3) = (-\widetilde{h}_2, \widetilde{h}_1, 0)
\quad\Rightarrow\quad
\begin{cases}
\displaystyle \frac{d\widetilde{h}_1}{dt} = -\widetilde{h}_2\\
\displaystyle \frac{d\widetilde{h}_2}{dt} = \widetilde{h}_1\\
\displaystyle \frac{d\widetilde{h}_3}{dt} = 0
\end{cases}
\quad\Rightarrow\quad
\begin{cases}
\displaystyle \frac{d^2\widetilde{h}_1}{dt^2} = -\frac{d\widetilde{h}_2}{dt} = -\widetilde{h}_1\\
\displaystyle \frac{d^2\widetilde{h}_2}{dt^2} =  \frac{d\widetilde{h}_1}{dt} = - \widetilde{h}_2\\
\displaystyle \frac{d^2\widetilde{h}_3}{dt^2} = 0
\end{cases}
\end{equation}
we then get 






