\begin{quote}
	Rmk:
For any subregion $W\subset D$, Let $W_t=\varphi_t(W)$, where $\varphi_t$ is the flow map.If the fluid is incompressible,
\begin{equation}
\frac{d}{dt}\int_{W_t} \rho dV = \int_{W_t}\frac{D\rho}{Dt} dV = 0,
\end{equation}
then
\begin{equation}
\int_{W_t} \rho(\textbf{x},t) dV_y = \int_{W}\rho(\varphi,t)J(\textbf{x},t)dV_x = \int_{W}\rho(\textbf{x},0) dV_x
\end{equation}
Now, we consider
\begin{equation}
0 = \frac{1}{\text{Volumn}(W)}\int_{W}\bigg(\rho\left(\varphi(\textbf{x},t),t\right)J(\textbf{x},t) - \rho(\textbf{x},0)\bigg) dV_x,\quad \forall W\subset D
\label{eq:unit volumn}
\end{equation}
Let $W=B(\textbf{x},r)$ and letting $r\to 0$, we have
\begin{equation}
\lim_{r\to 0}\frac{1}{\text{Volumn}(B(\textbf{x},r))}\int_{B(\textbf{x},r)}\bigg(\rho\left(\varphi(\textbf{x},t),t\right)J(\textbf{x},t) - \rho(\textbf{x},0)\bigg) dV_x.
\end{equation}
Yield that
\begin{equation}
\rho\left(\varphi(\textbf{x},t),t\right)J(\textbf{x},t) - \rho(\textbf{x},0) = 0.
\end{equation}
Hence
\begin{equation}
\rho\left(\varphi(\textbf{x},t),t\right) =  \rho(\textbf{x},0),\quad \forall \textbf{x}\in D, t>0
\end{equation}
if the fluid is incompressible.
\end{quote}



\begin{quote}
	Rmk:
\begin{itemize}
	\item Incompressible: $\rho\left(\varphi(\textbf{x},t),t\right) =  \rho(\textbf{x},0),\quad \forall \textbf{x}\in D, t>0$
	\item Homogeneous: $\rho (\textbf{x},t) = \rho(t),\quad\forall x\in D$

\end{itemize}

\end{quote}

\begin{quote}
	e.g.
For $\varphi(\textbf{x},t) = \varphi\left((x_1,x_2,x_3),t\right) = ((1+t)x_1,x_2,x_3)$, so the Jocobian
\begin{equation}
J(\textbf{x},t) = \begin{vmatrix}
1+t & 0 & 0\\
0 & 1 & 0\\
0 & 0 & 1\\
\end{vmatrix} = 1+t
\end{equation}
We can choose $\displaystyle \rho(\textbf{x},t) = \frac{1}{1+t}$, then the fluid is compressible but homogeneous.
\end{quote}



\subsubsubsection{Example} % H4 title

Consider an imcompressible homogeneous fluid in a regionthen the density  $\rho$ is a constant.

Then $\displaystyle \textbf{u}_t + (\textbf{u} \cdot \nabla ) \textbf{u} + \nabla p/\rho = 0$ (Euler's equation) in $D$, since it is incompressible, $\operatorname{div}\textbf{u} = 0$ in $D$​​. That is
\begin{equation}
\begin{cases}
\displaystyle \textbf{u}_t + (\textbf{u} \cdot \nabla ) \textbf{u} + \frac{\nabla p}{\rho} = 0\\
\operatorname{div} \textbf{u} = 0
\end{cases},\quad\text{in $D$}.
\end{equation}



Taking the derevitive of Euler's equation
\begin{equation}
\begin{aligned}
\operatorname{div}\left(\textbf{u}_t + (\textbf{u} \cdot \nabla) \textbf{u} + \frac{\nabla p}{\rho} \right) &= 0\\
\operatorname{div}(\textbf{u}_t) + \operatorname{div}\left((\textbf{u} \cdot \nabla) \textbf{u}\right) + \operatorname{div}\left(\frac{\nabla p}{\rho}\right) &= 0\\
\frac{\partial}{\partial t}\underbrace{\operatorname{div}(\textbf{u})}_{=0} + \operatorname{div}\left((\textbf{u} \cdot \nabla) \textbf{u}\right) + \operatorname{div}\left(\frac{\nabla p}{\rho}\right) &= 0\\
\operatorname{div}\left((\textbf{u} \cdot \nabla) \textbf{u}\right) + \operatorname{div}\left(\frac{\nabla p}{\rho}\right) &= 0\\
\end{aligned}
\end{equation}



