\subsubsubsection{2nd rank tensor} % H4 title

2nd rank tensor are matrices, a matrix $A = (A_{ij})$ is isotropic if for every rotation matrix $R = (R_{ij})$ we have
\begin{equation}
A_{ij} = R_{ik}R_{i\ell}A_{k\ell},\quad \forall i,j,=1,2,\ldots,n
\end{equation}
or in matrix from $RAR^{T} = A$. 

One can show that $n\geq 3$, $A=(A_{ij})$ is isotropic if and only if $A_{ij} = \lambda \delta_{ij}$  for soem constant $\lambda$, i.e.
\begin{equation}
A = \begin{pmatrix}
A_{1} & \cdots & 0\\
\vdots & \ddots & \vdots\\
0 & \cdots & A_{n}
\end{pmatrix}
\end{equation}


\subsubsubsection{3rd rank tensor} % H4 title

A 3rd rank tensor $A = (A_{ij})$ is isotropic if and only if for every rotation matrix $R_{ij}$, 
\begin{equation}
A_{ij} = R_{i\alpha}R_{j\beta}R_{k\gamma}\,A_{\alpha\beta\gamma}
\end{equation}

\vspace{5pt}
\hrule
\vspace{6pt}

Here, the stress tensor 
\begin{equation}
S_{ij} = -\delta_{ij}p + T_{ij}
\end{equation}
Since $T_{ij}$ is a linear function of $r_{ij}$, $T_{ij} = A_{ijk\ell}\,r_{k\ell}$ for 

We expect that $A_{ijk\ell}$ should be independent of rotation, hence, $A_{ijk\ell}$ should be as isotropic tensor. Then for isotropic $4$-tensors $A_{ijk\ell}$, one can prove that 
\begin{equation}
A_{ijk\ell} = a \delta_{ij}\delta_{k\ell}+ b \delta_{ik}\delta_{j\ell}+ c \delta_{i\ell}\delta_{jk},\quad a,b,c\in\mathbb{R}.
\end{equation}
Hence 
\begin{equation}
\begin{aligned}
T_{ij}
&= a \delta_{ij}\delta_{k\ell} r_{k\ell}
+ b \delta_{ik}\delta_{j\ell} r_{k\ell}
+ c \delta_{i\ell}\delta_{jk} r_{k\ell}\\
&= a\delta_{ij}\tr(R) + br_{ij}+cr_{ij}\\
&= a\delta_{ij}\tr(R) + (b+c)r_{ij},\quad\text{let $c_{1} = b+c$}\\
&= a\delta_{ij}\tr(R) + c_{1}r_{ij}
\end{aligned}
\end{equation}
Now, $\tr(R) = r_{kk} = \partial_{k}u_{k} = \operatorname{div}\textbf{u}$, substituting in, we habe
\begin{equation}
\begin{aligned}
T_{ij} &= a\delta_{ij}(\operatorname{div}\textbf{u}) + \frac{c_{1}}{2}
\left(\frac{\partial u_{i}}{\partial x_{j}} + \frac{\partial u_{j}}{\partial x_{i}}\right)\\
&= \mu \left(\frac{\partial u_{i}}{\partial x_{j}} + \frac{\partial u_{j}}{\partial x_{i}} - \frac{2}{3}(\operatorname{div}\textbf{u})\delta_{ij}\right) + \lambda (\operatorname{div}\textbf{i})\delta_{ij}
\end{aligned}
\end{equation}
where $\lambda = a + 2\mu/3$, $\mu = c_2/2$.

Since $\displaystyle 2r_{ij} = \frac{\partial u_{i}}{\partial x_{j}} + \frac{\partial u_{j}}{\partial x_{i}}$​

\begin{enumerate}
	\item $$
    \begin{aligned}    \sum\textit{{i=1}^{3}\frac{\partial}{\partial x}{i}} \left(2\mu r_{ij}\right)    &= \mu \sum\textit{{i=1}^{3}\frac{\partial}{\partial x}{i}}\left(\frac{\partial u\textit{{i}}{\partial x}{j}} + \frac{\partial u\textit{{j}}{\partial x}{i}}\right)\\    &= \mu \sum\textit{{i=1}^{3}\frac{\partial^2 u}{i}}{\partial x\textit{{j}^2} + \mu \sum}{i=1}^{3}\frac{\partial^2 u\textit{{j}}{\partial x}{j}\partial x_{i}}\\    &= \mu \nabla^2 u\textit{{j} + \mu \frac{\partial}{\partial x}j}(\operatorname{div}\textbf{u})    \end{aligned}
\begin{equation}
2. $(\operatorname{div}\textbf{u})\delta_{ij} = \begin{cases}0 & \text{if $i\neq j$}\\\operatorname{div}\textbf{u}&\text{if $i=j$}\end{cases}$.

So that
\end{equation}
\sum\textit{{i=1}^{3} \frac{\partial}{\partial x}{i}}\delta\textit{{ij} \operatorname{div}\textbf{u} =  \frac{\partial}{\partial x}{j}}(\operatorname{div}\textbf{u})
\begin{equation}
then
\end{equation}
\begin{aligned}\sum\textit{{i=1}^{3} \frac{\partial}{\partial x}{i}}\left(-\frac{2\mu}{3}(\operatorname{div}\textbf{u}) \delta\textit{{ij} + \lambda (\operatorname{div}\textbf{u})\delta}{ij}\right)&= \left(\lambda - \frac{2\mu}{3}\right) \sum\textit{{i=1}^{3} \frac{\partial}{\partial x}{i}}\left((\operatorname{div}\textbf{u})\delta_{ij}\right)\\&= \left(\lambda - \frac{2\mu}{3}\right)\frac{\partial}{\partial x_{j}}(\operatorname{div}\textbf{u})\end{aligned}
\begin{equation}
combinde all
\end{equation}
\begin{aligned}\sum\textit{{i=1}^{3}\frac{\partial}{\partial x}{i}} \left[2\mu \left(r\textit{{ij} - \frac{1}{3}(\operatorname{div}\textbf{u})\delta}{ij}\right)\right] + \lambda (\operatorname{div}\textbf{u})\delta_{ij}&= \mu \nabla^2 u\textit{{j} + \left(\lambda + \frac{\mu}{3}\right)\frac{\partial}{\partial x}j}(\operatorname{div}\textbf{u})\end{aligned}
\begin{equation}
then the force per unit area due to shear stress tensor in the $j$-th direction
\end{equation}
\begin{aligned}\frac{\delta F_{j}}{\delta V}&= \frac{\partial}{\partial x\textit{{i}}\left(-\delta}{ij}p + T_{ij}\right)\\&= \frac{\partial p}{\partial x_{j}} + \frac{\partial}{\partial x\textit{{i}}\left[2\mu \left(r}{ij} - \frac{1}{2}(\operatorname{div}\textbf{u})\delta\textit{{ij}\right) + \lambda(\operatorname{div}\textbf{u})\delta}{ij}\right]\\&= \frac{\partial p}{\partial x\textit{{j}} + \mu\frac{\partial}{\partial x}{i}}\left(\frac{\partial u\textit{{i}}{\partial x}{j}} + \frac{\partial u\textit{{j}}{\partial x}{i}}\right) + \left(\lambda - \frac{2}{3}\mu\right)\frac{\partial}{\partial x_{j}}(\operatorname{div}\textbf{u})\\&= \frac{\partial p}{\partial x\textit{{j}} + \mu \nabla^2 u}{j} + \mu \frac{\partial}{\partial x\textit{{j}}(\operatorname{div}\textbf{u}) + \left(\lambda - \frac{2}{3}\mu\right)\frac{\partial}{\partial x}{j}}(\operatorname{div}\textbf{u})\\&= \frac{\partial p}{\partial x\textit{{j}} + \mu \nabla^2 u}j + \left(\lambda - \frac{1}{3}\mu\right)\frac{\partial}{\partial x_{j}}(\operatorname{div}\textbf{u})\end{aligned}
\begin{equation}
Last, we have
\end{equation}
\begin{aligned}\textbf{u}_{t} + \left(\textbf{u}\cdot\nabla\right)\textbf{u} &= -\frac{\nabla p}{\rho} + \frac{1}{\rho}\nabla\cdot T + \vec{b}\\&= -\frac{\nabla p}{\rho} + \frac{1}{\rho}\left(\mu \nabla^2\textbf{u} + \left(\lambda + \frac{1}{3}\mu\right)\nabla(\operatorname{div}\textbf{u})\right) + \vec{b}\end{aligned}
\begin{equation}
which is *Navier-Stoke's equation* or
\end{equation}
\rho \frac{D\textbf{u}}{Dt} = -\nabla p + \left(\mu \nabla^2\textbf{u} + \left(\lambda + \frac{1}{3}\mu\right)\nabla(\operatorname{div}\textbf{u})\right) + \rho\vec{b}
\begin{equation}
> For incompressible homogeneouse flow ( $\rho = \rho_0$ is a constant ) without external force, the EoM is
>
\end{equation}

\rho_0 \frac{D\textbf{u}}{Dt} = -\nabla p + \mu \nabla^2 \textbf{u}
\begin{equation}
or
\end{equation}
\frac{D\textbf{u}}{Dt} = -\nabla p' + \nu \nabla^2 \textbf{u}
\begin{equation}
where $p' = p/\rho_0$ and
\end{equation}
\nu = \frac{\mu}{\rho}
\end{enumerate}
