
% Sets the document class and font size
\documentclass[12pt]{article}
\usepackage[a4paper, left=1in, right=1in, top=0.8in, bottom=0.9in]{geometry}
\let\mathbbalt\mathbb
\usepackage[utf8]{inputenc}    % Input encoding
\usepackage[T1]{fontenc}       % Font encoding
\usepackage{fontspec}          % font encoding



% Advanced math typesetting
\usepackage{amsmath}
\usepackage{amssymb}
\usepackage{mathtools}
\usepackage{physics}


% Symbols and Text
\usepackage{bbold}     % bold font
\usepackage{ulem}      % strikethrough
\usepackage{listings}  % Source code listing
\usepackage{import}    % Importing code and other documents

% graphics
\usepackage[dvipsnames]{xcolor}
\usepackage{graphicx}
\usepackage{array}
\usepackage{tikz}
\usepackage{pgfplots}
\usepackage{tcolorbox}
%\usepackage{changepage}

% figure
\usepackage{float}
\usepackage{subfigure}

\usepackage{hyperref}  % Hyperlinks in the document
\hypersetup{
    colorlinks=true,
    linkcolor=Blue,
    filecolor=red,
    urlcolor=Blue,
    citecolor=blue,
    pdftitle={Article},
    pdfauthor={Author},
}

\usepackage{xeCJK}               % Chinese, Japanese, and Korean characters
\setCJKfamilyfont{kai}{標楷體}    % Chinese font

\usepackage{subfiles}  % Best loaded last in the preamble
\usepackage{titlesec}  % fortitle

\titleclass{\subsubsubsection}{straight}[\subsection]

\newcounter{subsubsubsection}[subsubsection]
\renewcommand\thesubsubsubsection{\thesubsubsection.\arabic{subsubsubsection}}
\renewcommand\theparagraph{\thesubsubsubsection.\arabic{paragraph}} % optional; useful if paragraphs are to be numbered

\titleformat{\subsubsubsection}
  {\normalfont\normalsize\bfseries}{\thesubsubsubsection}{1em}{}
\titlespacing*{\subsubsubsection}
{0pt}{3.25ex plus 1ex minus .2ex}{1.5ex plus .2ex}

\makeatletter
\renewcommand\paragraph{
    \@startsection{paragraph}{5}{\z@}{3.25ex\@plus1ex\@minus.2ex}{-1em}{\normalfont\normalsize\bfseries}
}
\renewcommand\subparagraph{
    \@startsection{subparagraph}{6}{\parindent}{3.25ex\@plus1ex\@minus.2ex}{-1em}{\normalfont\normalsize\bfseries}
}
\def\toclevel@subsubsubsection{4}
\def\toclevel@paragraph{5}
\def\toclevel@paragraph{6}
\def\l@subsubsubsection{\@dottedtocline{4}{7em}{4em}}
\def\l@paragraph{\@dottedtocline{5}{10em}{5em}}
\def\l@subparagraph{\@dottedtocline{6}{14em}{6em}}
\makeatother

\setcounter{secnumdepth}{4}
\setcounter{tocdepth}{4}

\usepackage{fancyhdr}
\pagestyle{fancy}

\title{Note: Vector Calculus}
\date{\today}
\author{Chang-Mao Yang 楊長茂}

\begin{document}
%================================================================================================
\maketitle

\subsection{coordinate convention}
\begin{figure}[htp]
\subfile{img/coordinate.tex}
\end{figure}



\subsection{unit vector(basis) transformation}
\begin{table}[htp]
\caption{Basis Transformation}

\begin{center}
\makebox[\textwidth]{
\begin{tabular}{c|c|c|c}
 	& Cartesian & Cylindrical & Spherical\\\hline 
\raisebox{-1.85em}{\rotatebox{90}{ Cartesian }}
& $\begin{aligned}
	\hat{e}_{x} &= \hat{e}_{x} \\
	\hat{e}_{y} &= \hat{e}_{y} \\ 
	\hat{e}_{z} &= \hat{e}_{z}
	\end{aligned}$
& $\begin{aligned}
	\hat{e}_{x} &= \cos\theta \,\hat{e}_{r} - \sin\theta \,\hat{e}_{\theta} \\
	\hat{e}_{y} &= \sin\theta \,\hat{e}_{r} + \cos\theta \,\hat{e}_{\theta} \\ 
	\hat{e}_{z} &= \hat{e}_{z}
	\end{aligned}$
& $\begin{aligned}
	\hat{e}_{x} &= \sin\theta\cos\varphi\,\hat{e}_{r} 
					+ \cos\theta\cos\varphi\,\hat{e}_{\theta} 
					- \sin\varphi\,\hat{e}_{\phi}\\
	\hat{e}_{y} &= \sin\theta\sin\varphi\,\hat{e}_{r} 
					+ \cos\theta\sin\varphi\,\hat{e}_{\theta} 
					+ \cos\varphi\,\hat{e}_{\phi}\\ 
	\hat{e}_{z} &= \cos\theta\,\hat{e}_{r} - \sin\theta\,\hat{e}_{\theta}
	\end{aligned}$
\\\hline
\raisebox{-2.15em}{\rotatebox{90}{Cylindrical}}
& $\begin{aligned}
	\hat{e}_{r} &= \frac{x\hat{e}_x + y\hat{e}_y}{\sqrt{x^2+y^2}} \\
	\hat{e}_{\theta} &= \frac{-y\hat{e}_x + x\hat{e}_y}{\sqrt{x^2+y^2}} \\ 
	\hat{e}_{z} &= \hat{e}_{z}
	\end{aligned}$
& $\begin{aligned}
	\hat{e}_{r} &= \hat{e}_{r} \\
	\hat{e}_{\theta} &= \hat{e}_{\theta} \\ 
	\hat{e}_{z} &= \hat{e}_{z}
	\end{aligned}$
& $\begin{aligned}
	\hat{e}_{r} &= \sin\theta\,\hat{e}_{r}+\cos\theta\,\hat{e}_{\theta} \\
	\hat{e}_{\theta} &= \hat{e}_{\varphi} \\ 
	\hat{e}_{z} &= \cos\theta\,\hat{e}_{r}-\sin\theta\,\hat{e}_{\theta}
	\end{aligned}$
\\\hline
\raisebox{-1.6em}{\rotatebox{90}{Spherical}}
& $\begin{aligned}
	\hat{e}_{r} &= \frac{x\hat{e}_{x} + y\hat{e}_{y} + z\hat{e}_{z}}{\sqrt{x^2+y^2+z^2}} \\
	\hat{e}_{\theta} &= \frac{z\left(x\hat{e}_{x} + y\hat{e}_{y}\right)-\left(x^2+y^2\right)\hat{e}_{z}}{\sqrt{x^2+y^2+z^2}\sqrt{x^2+y^2}}  \\ 
	\hat{e}_{\varphi} &= \frac{-y\hat{e}_x + x\hat{e}_y}{\sqrt{x^2+y^2}}
	\end{aligned}$
& $\begin{aligned}
	\hat{e}_{r} &= \frac{r\hat{e}_r + z\hat{e}_z}{\sqrt{r^2+z^2}} \\
	\hat{e}_{\theta} &= \frac{z\hat{e}_r - r\hat{e}_z}{\sqrt{r^2+z^2}} \\ 
	\hat{e}_{\varphi} &= \hat{e}_{\varphi}
	\end{aligned}$
& $\begin{aligned}
	\hat{e}_{r} &= \hat{e}_{r} \\
	\hat{e}_{\theta} &= \hat{e}_{\theta} \\ 
	\hat{e}_{\varphi} &= \hat{e}_{\varphi}
	\end{aligned}$
\end{tabular}}
\end{center}
\label{table:basis-transformation}
\end{table}
















%================================================================================================
\end{document}