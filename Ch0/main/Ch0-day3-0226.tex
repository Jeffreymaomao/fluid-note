\subsubsection{Lemma} % H3 title

Let $D$ be a bounded domain and $F: \bar{D}\times[0,a_0]\to \mathbb{R}$ be a smooth function (or $C^{\infty}$), then 
\begin{equation}
\frac{d}{dt}\int_{D} F(x,t) dx = \int_D \frac{dF(x,t)}{dt}dx
\end{equation}


\subsubsection{proof:} % H4 title

we have
\begin{equation}
\begin{aligned}
\frac{d}{dt} \int_D F(x,t) dx
&= \lim_{\Delta t\to 0}\left[
\frac{1}{\Delta t} \int_D F(x,t+\Delta t) dx
- \frac{d}{dt} \int_D F(x,t) dx
\right]\\
&= \lim_{\Delta t\to 0} \frac{d}{dt} \int_D \frac{F(x,t+\Delta t) - F(x,t)}{\Delta t} dx\\
&= \text{By M.V.T.}\\
&= \lim_{\Delta t\to 0} \int_D \frac{\frac{\partial}{\partial t} F(x,\xi) \Delta t}{\Delta t} dx,\quad \text{for some $\xi$, where} t<\xi<t+\Delta\\
&= \lim_{\Delta t\to 0}\int_{D} \frac{\partial}{\partial t}F(x,\xi) dx
\end{aligned}
\end{equation}
Denote, $\displaystyle \frac{\partial}{\partial t}F(x,t) = F_t(x,t)$ and $\displaystyle \frac{\partial^2}{\partial t^2}F(x,t) = F_{tt}(x,t)$, so


\begin{equation}
\begin{aligned}
\quad&
\abs{\frac{1}{\Delta t}\int_D [F(x,t+\Delta t) - F(x,t)] - \int_D \frac{\partial}{\partial t}F(x,t)dx} \\
&= \abs{\int_{D}F_t(x,\xi) dx - \int_{D}F_t(x,\xi)dx}\\
&= \text{By MVT}\\
&= \abs{\int_{D}[F_t(x,\xi) - F_t(x,t)]}dx\\
&= \text{By MVT}\\
&= \abs{\int_{D} F_{tt}(x,z)(t-\xi)dz} ,\quad\text{$z$ between $t$ and $\xi$}\\
&\leq M\abs{t-\xi}\abs{D}\to 0,\quad \text{where $\abs{D}$ is vlolumn of $D$}
\end{aligned}
\end{equation}
where $\displaystyle M=\sup_{(x,t)\in D\times (0,a)} F_{tt}(x,t)$.

\subsubsection{The Continuity Equation} % H3 title

Recall that $D$ is a region in $\mathbb{R}^2$ or $\mathbb{R}^3$ filled with fluid and $x=x(x_1,x_2,x_3)\in D$ is a particle of fluid moving through $x$ at time $t$ with velocity $\textbf{u}(x,t)$. If $W$ is a ant subregion of $D$, then  the mass of fluid in $W$ at time $t$ is 
\begin{equation}
m(W,t) = \int_{W} \rho(x,t)dV,
\end{equation}
where $\rho(x,t)$ is the density of fluid at $(x,t)$. Then $\displaystyle \frac{d}{dt}m(W,t) = \int_W \rho_t (x,t) dV$​.

\underline{Fig3 - boundary of W}

Let $\partial W$ be the boundary of $W$. Suppose $\partial W$ is smooth, let $\textbf{n}(x)$ be the normal vector to $\partial W$ at $x\in \partial W$, Let $dA$ denote the 

Then the volumn of fluid flow rate across $\partial W$ per unit time, Since $\textbf{u}\cdot \textbf{n}\, \Delta A$$\Rightarrow$ the mass of fluid flow per unit time is $\rho \textbf{u}\cdot \textbf{n}\, \Delta A$

Since, by the conservation of mass, the rate of increase of mass in $W$ is equal to the rate that mass is incoming $\partial W$
\begin{equation}
\int_{W} \rho_t  dV
= \frac{d}{dt} \int_{W} \rho  dV
= - \int_{\partial W} \rho \,(\textbf{n}\, \cdot \textbf{u}) dV
= \text{by divergence theorem}
= - \int_{W} \operatorname{div}(\rho\textbf{u})dV
\end{equation}
By divergence theorem, we have
\begin{equation}
\int_{W}\left(\rho_t + \operatorname{div}(\rho\textbf{u})\right) dV = 0,\quad\forall W\subset D
\end{equation}
We now choose $W=B(x,r)$, and let $H(y,t) =\rho_t + \operatorname{div}(\rho\textbf{u})=H$, by above equation, we have
\begin{equation}
\int_{B(x,r)} H(y,t) dV_x = 0,\quad \forall x\in D, B(x,r)\subset D\label{eq:H}
\end{equation}
Notice $\displaystyle H(y,t) = \frac{1}{\abs{B(x,r)}}\int_{B(x,r)} H(y,t) dV = H(y,t)\frac{\int_{B(x,r)}dV}{\abs{B(x,r)}}$, where $\abs{B(x,r)}$ is the volumn of $B(x,r)$. Now 
\begin{equation}
\begin{aligned}
\abs{\frac{1}{B(x,r)} \int_{B(x,r)} H(x,t)dV - H(x,t)}
&= \frac{1}{\abs{B(x,r)}} \abs{\int_{B(x,r)} [H(y,t) - H(x,t)]dV } \\
&\leq \frac{1}{\abs{B(x,r)}} \int_{B(x,r)} \abs{H(y,t) - H(x,t)}dV  \\
&\leq \frac{1}{\abs{B(x,r)}} \max_{y\in B(x,r)} \abs{H(y,t) - H(x,t)} \cdot \abs{B(x,r)} \\
&\to 0,\quad\text{as } r \to 0
\end{aligned}
\end{equation}
so that 
\begin{equation}
\lim_{r\to 0}\abs{\frac{1}{B(x,r)} \int_{B(x,r)} H(x,t)dV - H(x,t)} = 0\label{eq:tozero}
\end{equation}
By equation $(\ref{eq:H})$ and $(\ref{eq:tozero})$, we have
\begin{equation}
H(x,t)=0\quad\Rightarrow\quad \rho_t + \operatorname{div}(\rho\textbf{u}) = 0
\end{equation}
which is called the continuity equation in $D\times (0,T)$.






