Consider a a tagged (marked) portion $\Sigma$ of fluid with center of mass at $(x_1(s), x_2(s), x_3(s))$ at time $s$. Let $m(x)$ and $V(s)$ be the mass and volumn of this portion $\Sigma$ of fluid at time $s$.The portion of fluid particle moves aling with fluid. see as (2/21 fig1)

For a time $t>0$, suppose at time $t $, the tagged portion $\Sigma$ of fluid particles is a cube centered at $(x_1, x_2, x_3)$ with side lengh $\ell_1,\ell_2,\ell_3$​, see as (2/21 fig2 - textbook p.4),where 
\begin{equation}
\begin{aligned}
P_1(s) &= \left(x_1(s)-\frac{\ell_1(s)}{2},x_2(s),x_3(s)\right)\\
P_2(s) &= \left(x_1(s)+\frac{\ell_1(s)}{2},x_2(s),x_3(s)\right)\\
\end{aligned}
\end{equation}



We assume that $\Sigma$ remain a cube for $s\approx t$ with side length, $\ell_1(s),\ell_2(s),\ell_3(s)$, then $V(s) = \ell_1(s)\cdot\ell_2(s)\cdot\ell_3(s)$
\begin{equation}
\frac{dV(s)}{ds}\Bigg|_{s=t}
=   \frac{d\ell_1(s)}{ds}\Bigg|_{s=t} \ell_2(s)\ell_3(s)
+\frac{d\ell_2(s)}{ds}\Bigg|_{s=t} \ell_1(s)\ell_3(s)
+\frac{d\ell_3(s)}{ds}\Bigg|_{s=t} \ell_1(s)\ell_2(s)
\end{equation}
where
\begin{equation}
\begin{aligned}
\frac{d\ell_1(s)}{ds}
&= u_1(P_2(t),t) - u_1(P_1(t),t)\\
&= u_1\left(x_1(s)+\frac{\ell_1(s)}{2},x_2(s),x_3(s),t\right)
- u_1\left(x_1(s)-\frac{\ell_1(s)}{2},x_2(s),x_3(s),t\right)\\
&\approx \frac{\partial u_1}{\partial x_1}\left(x_1,x_2,x_3,t\right)\cdot \ell_1
\end{aligned}
\end{equation}
Similarly 
\begin{equation}
\frac{d\ell_i}{ds}\Bigg|_{s=t} = \frac{\partial u_i}{\partial x_i}\left(x_1,x_2,x_3\right)\cdot \ell_{i},\quad \forall i=1,2,3.
\end{equation}
Now we write $\displaystyle \frac{d}{ds}\Bigg|_{s=t}=\frac{d}{dt}$​.

\underline{Fig2 = tagged portion and face}

\subsubsection{Continuity equation} % H3 title

Let $\rho(\textbf{x},t)$ be the density of fluid at time $s$.Since $M(s)=\operatorname{const.}$, $\forall s>0$ and $\displaystyle \frac{dM(s)}{ds} = 0$, $\forall s>0$.Therefore, since it is similar to the cube, the density is
\begin{equation}
\rho(\textbf{x},s) \approx \frac{M(s)}{V(s)}
\end{equation}
and the derevative is 
\begin{equation}
\begin{aligned}
\frac{d}{ds}\rho(\textbf{x},s) \Bigg|_{s=t}
&\approx \frac{d}{ds}\frac{M(s)}{V(s)} \Bigg|_{s=t}\\
&= \frac{M'(s)V(s) - M(s)V'(s)}{V^2(s)} \Bigg|_{s=t}\\
&= \frac{0 - M(s)\frac{d}{ds}V(s)}{V^2(s)} \Bigg|_{s=t}\\
&=  -\frac{M(s)(\operatorname{div}\textbf{u})V(s)}{V^2(s)} \Bigg|_{s=t}\\
&=  -\frac{M(s)}{V(s)}(\operatorname{div}\textbf{u}(s)) \Bigg|_{s=t}\\
&=  -\rho(\textbf{x}(s),s)(\operatorname{div}\textbf{u}(s)) \Bigg|_{s=t}\\
\end{aligned}
\end{equation}
we get 
\begin{equation}
- \frac{d}{dt}\rho(\textbf{x}(t),t) = \rho \cdot (\nabla\cdot \textbf{u}(t))
\end{equation}
On the other hand, by chain rule
\begin{equation}
\frac{d}{dt}\rho(\textbf{x}(t),t) = \rho_t + (\nabla\rho)\cdot \textbf{u}(t)
\end{equation}
combining together we have
\begin{equation}
\begin{aligned}
&\Rightarrow \rho_t + (\nabla\rho)\cdot \textbf{u} = \rho \cdot (\nabla\cdot \textbf{u})\\
&\Rightarrow \rho_t + (\nabla\rho)\cdot \textbf{u} - \rho \cdot (\nabla\cdot \textbf{u}) = 0\\
&\Rightarrow \rho_t + \nabla\cdot (\rho\textbf{u} ) = 0
\end{aligned}
\end{equation}
and the equation $\rho_t + \nabla\cdot (\rho\textbf{u} ) = 0$ is called the \textit{contunuity equation.}

\subsubsection{Heuristic proof of the Euler equation} % H3 title

In the ansense of an externally applied forces, the net force $\textbf{F}$, acting on $\delta V$, is due to the pressure field.

Write $\textbf{F}=(F_1,F_2,F_3)$, we get
\begin{equation}
\begin{aligned}
\textbf{F}(x_1,x_2,x_3,t)
&\approx \left(
P\left(x_1-\frac{\delta x_1}{2},x_2,x_3,t\right)
- P\left(x_1+\frac{\delta x_1}{2},x_2,x_3,t\right)
\right)\delta x_2 \delta x_3\\
&= -\frac{\partial P}{\partial x_1}(\zeta_{1},x_2,x_3, t)\delta x_1 \delta x_2 \delta x_3,\quad \delta x_1, \delta x_2, \delta x_3\to 0\\
&= \frac{\partial P}{\partial x_1}(\zeta_{1},x_2,x_3, t)\delta V
\end{aligned}
\end{equation}
for some $\zeta_1\in\left(x_1-\frac{\delta x_1}{2}, x_1+\frac{\delta x_1}{2}\right)$.

By Newton's second law, the equation of motion for the elemnet of fund mass $\delta m$, at point $\textbf{x}(t)$ is 
\begin{equation}
\frac{d}{dt}\left(\delta m \cdot \textbf{u}(\textbf{x},t)\right) = \textbf{F} = -(\nabla P)\delta V
\end{equation}
also
\begin{equation}
\frac{d}{dt}\left(\delta m \cdot \textbf{u}(\textbf{x},t)\right) = \delta m \frac{d}{dt}\textbf{u}(\textbf{x},t) = \delta m \left(\textbf{u}_t+(\nabla\cdot \textbf{u})\right)\,\textbf{u}
\end{equation}
then
\begin{equation}
\begin{aligned}
\delta m \left(\textbf{u}_t+(\nabla\cdot \textbf{u})\right) \,\textbf{u}
&=
-(\nabla P)\delta V\\
\textbf{u}_t+ (\nabla\cdot \textbf{u})\,\textbf{u} &= -(\nabla P)\frac{\delta V}{\delta m} = -(\nabla P)\frac{1}{\delta m/\delta V}
\end{aligned}
\end{equation}
we get a equation
\begin{equation}
\textbf{u}_t+ (\nabla\cdot \textbf{u}) \,\textbf{u} = -\frac{\nabla P}{\rho}
\end{equation}


called \textit{Euler's equation}.

\begin{quote}
	Notice that
\begin{equation}
\left(\nabla\cdot \textbf{u}\right)\,\textbf{u} = \left(\sum_{i=0}^{3}u_i\frac{\partial}{\partial x_i}\right)\textbf{u}
= \left(u_1\frac{\partial}{\partial x_1}
+u_2\frac{\partial}{\partial x_2}
+u_3\frac{\partial}{\partial x_3}\right)
\begin{pmatrix}u_1\\ u_2\\ u_3\\
\end{pmatrix}
=\begin{pmatrix}
\left(u_1\frac{\partial}{\partial x_1}
+u_2\frac{\partial}{\partial x_2}
+u_3\frac{\partial}{\partial x_3}\right) u_1\\
\left(u_1\frac{\partial}{\partial x_1}
+u_2\frac{\partial}{\partial x_2}
+u_3\frac{\partial}{\partial x_3}\right) u_2\\
\left(u_1\frac{\partial}{\partial x_1}
+u_2\frac{\partial}{\partial x_2}
+u_3\frac{\partial}{\partial x_3}\right) u_3\\
\end{pmatrix}
\end{equation}

\end{quote}
