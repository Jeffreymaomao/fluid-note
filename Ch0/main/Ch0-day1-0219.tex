\section{The Equation of Motion} % H1 title

\subsection{Introduction} % H2 title

\subsubsection{Euler's equation:} % H3 title

Consider a fluid in a domain $D$ in $\mathbb{R}^{n}$  ( $n=2$, $n=3$ ).

Let $x\in D$, and $\rho(\textbf{x},t)$, $\textbf{u}(\textbf{x},t)$, $p(\textbf{x},t)$ be the fluid density, velocity vector field and the pressure at the point $x$ and time $t$.Consider an infinitesimal element of the fluid of volumn $\partial V$ located at point $x$ at time $t$ with mass $\delta m = \rho(\textbf{x},t)$, which is moving $\textbf{u}(\textbf{x},t)$ and momentum $\delta m \cdot \textbf{u}(\textbf{x},t)$

The normal force directed into the indeinetesmal volumn across a face of area $\delta a$ is $\textbf{n}\cdot p(\textbf{x},t)\cdot \delta a$

\begin{center}\subfile{img/Fig1-box.tex}\end{center}

Note that the pressure is the magnitade of the torce per unit area or normal stress, imposed on the fluid from neighboring fluid elements.

\subsubsection{Convective derivative} % H3 title
convective derivative 對流導數 / material derivative 物質導數 / advective derivative 隨流導數 / convective derivative 對流導數 / derivative following the motion 隨體導數 / hydrodynamic derivative 水動力導數 / Lagrange derivative 拉格朗日導數 / substantial derivative 隨質導數Couvder a fluid particle moving in flaid, whose position $\textbf{x}$ at time $t$ is $\textbf{x}(t)$. Then 
\begin{equation}
\frac{d\textbf{x}(t)}{dt} = \dot{\textbf{x}} = \textbf{u}(\textbf{x}(t),t)
\end{equation}
Hence, if $f(\textbf{x},t)$ is a function on $D\times (0,T)$, then $f(\textbf{x}(t), t)$ is the value if $f$ at the moving fluid particle at $\textbf{x}(t)$ at time $t$. We define the convective derivative of $f$:
\begin{equation}
\begin{aligned}
\frac{Df(\textbf{x},t)}{Dt} &= \frac{\partial f(\textbf{x},t)}{\partial t} + \dot{\textbf{x}} \cdot \nabla f(\textbf{x},t)\\
&= f_t + \textbf{u}\cdot \nabla f
\end{aligned}
\end{equation}


where $\nabla f = f(f_x,f_y,f_z)$ and $\textbf{u}=(u_1,u_2,u_3)$.

Hence, if $f(\textbf{x},t)$ is a function on $D\times(0,T)$, then $f(\textbf{x}(t),t)$ is the vlaue of $f$ at the moving fluid particle at $\textbf{x}(t)$ at time $t$.

We define the convective derivative of $f$ as:
\begin{equation}
\begin{aligned}
\frac{Df(x,t)}{Dt}
&= \frac{\partial f}{\partial t} + \dot{\textbf{x}}(t)\cdot \nabla f\\
&= f_t + \textbf{u} \cdot \nabla f
\end{aligned},
\end{equation}
 where $\nabla f = (f_x,f_y,f_z)$ and $\textbf{u} = (u_1,u_2,u_3)$.

\subsubsubsection{Def.} % H4 title

For any vector filed $\textbf{F} = (F_1,F_2,\ldots,F_n)$ on $D$, we let
\begin{equation}
\int_{D} \textbf{F}dV = \left(\int_{D} F_1dV, \int_{D} F_2dV,\ldots,\int_{D} F_ndV\right).
\end{equation}


\subsubsubsection{Def.} % H4 title

We will assume that $D$ is a smooth domain, i.e. for any $x_0 \in \partial D$, $\mathbb{R}^{n} = (x',x_n), n=2,3$

$\exists \delta_0 > 0$ and a smooth function $\varphi:\mathbb{R}^{n-1}\to\mathbb{R}$, s.t.
\begin{equation}
\partial D\cap B(x_0,\delta_0)=\left\{(x',\varphi(x')) : \norm{x'}<\delta_0, x'\in\mathbb{R}^{n-1}\right\} \cap B(x_0,\delta_0)
\end{equation}
and
\begin{equation}
D\cap B({x_0,\delta_0}) = \left\{(x',x_n) : x_n>\varphi(x'), x'\in\mathbb{R}^{n-1}, \norm{x'}<\delta_0\right\} \cap B(x_0,\delta_0)
\end{equation}


\subsubsubsection{Claim} % H4 title

Conside the volume $\delta V$ of an element of mass $\delta m$, which moves in the fluid by the fluid motion
\begin{equation}
\frac{d(\delta V)}{dt} = (\nabla \cdot \textbf{u}) (\textbf{x},t)\cdot \delta V\quad\text{as}\quad \delta x_1, \delta x_2, \delta x_3 \to 0,
\end{equation}
 where $\displaystyle \nabla\cdot\textbf{u} = \operatorname{div}\textbf{u} = \sum_{i=1}^{3}\frac{\partial u_i}{\partial x_i}$, $\textbf{u} = (u_1,u_2,u_3)$.

\subsubsubsection{proof} % H4 title


\begin{equation}
\begin{aligned}
\frac{d(\delta V)}{dt}
&= \frac{d}{dt}(\delta x_1, \delta x_2, \delta x_3)\\
&= \frac{d(\delta x_1)}{dt} \delta x_2 \delta x_3
+ \frac{d(\delta x_2)}{dt} \delta x_1 \delta x_3
+ \frac{d(\delta x_3)}{dt} \delta x_1 \delta x_2
\end{aligned}
\end{equation}

For the first term
\begin{equation}
\begin{aligned}
\frac{d(\delta x_1)}{dt}
&\approx u_1\left(x_1+\frac{\delta x_1}{2}, x_2, x_3\right)
- u_1\left(x_1-\frac{\delta x_1}{2}, x_2, x_3\right)\\
&= \frac{\partial u_1}{\partial x_1}(\xi_1, x_2,x_3) \delta x_1,\quad \text{fot some $\xi_1\in\left(x_1-\frac{\delta x_1}{2},x_1+\frac{\delta x_1}{2}\right)$}
\end{aligned}
\end{equation}
then 
\begin{equation}
\frac{d(\delta x_1)}{dt}\delta x_2 \delta x_3 \to
\frac{\partial u_1}{\partial x_1}(x_1, x_2,x_3) \delta x_1\delta x_2 \delta x_3,
\quad \text{as $\delta x_1,\delta x_2, \delta x_3 \to 0$}
\end{equation}
Similarly 
\begin{equation}
\frac{d(\delta x_2)}{dt}\delta x_2 \delta x_3 \to
\frac{\partial u_2}{\partial x_1}(x_1, x_2,x_3) \delta x_1\delta x_2 \delta x_3,
\quad \text{as $\delta x_1,\delta x_2, \delta x_3 \to 0$}
\end{equation}
and 
\begin{equation}
\frac{d(\delta x_3)}{dt}\delta x_2 \delta x_3 \to
\frac{\partial u_3}{\partial x_1}(x_1, x_2,x_3) \delta x_1\delta x_2 \delta x_3,
\quad \text{as $\delta x_1,\delta x_2, \delta x_3 \to 0$}
\end{equation}
so that 
\begin{equation}
\frac{d(\delta V)}{dt} = \left(
\frac{\partial u_1}{\partial x_1}
+ \frac{\partial u_2}{\partial x_2}
+ \frac{\partial u_3}{\partial x_3}
\right) \delta x_1\delta x_2 \delta x_3
= (\nabla \cdot \textbf{u}) \delta V
\end{equation}

